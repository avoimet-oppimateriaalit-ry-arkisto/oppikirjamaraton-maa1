\newpage

\section*{Metasivu}

Projektin nimi: Oppikirjamaraton

Kirjan LaTeX-lähdekoodi: \\
\url{https://github.com/linjaaho/oppikirjamaraton-maa1}

Versio: 0.92 \qquad lähdekoodi ajettu \today \\
Ensimmäinen julkaistu versio (0.9) kirjoitettiin viikonlopun 28.--30.9.2012 aikana. Kirjaa on jatkokehitetty sen jälkeen ja uusi versio julkaistaan aina, kun edistystä on tapahtunut riittävästi.

Lisenssi: CC BY 3.0 (\url{http://creativecommons.org/licenses/by/3.0/legalcode})\\
Sinulla on vapaus:
\begin{enumerate}
\item jakaa — kopioida, levittää, näyttää ja esittää teosta
\item remiksata — valmistaa muutettuja teoksia
\item käyttää teosta kaupallisiin tarkoituksiin
\end{enumerate}
Seuraavilla ehdoilla:
\begin{enumerate}
\item nimeä — Teoksen tekijä on ilmoitettava siten kuin tekijä tai teoksen lisensoija on sen määrännyt (mutta ei siten että ilmoitus viittaisi lisenssinantajan tukevan lisenssinsaajaa tai Teoksen käyttötapaa)
\end{enumerate}

Määräys lisäyksenä lisenssiin: kaikkien tekijöiden nimet on ilmoitettava jossakin kohtaa kirjaa. Nimet löytyvät liitteestä G.

\subsection*{Tue tekijöitä ja avointen oppimateriaalien luomista!}

\subsubsection*{Flattr}

\includegraphics[scale=0.2]{MAA1-Flattr.png} \\
\url{http://flattr.com/t/914482}

\subsubsection*{Bitcoin}

\includegraphics[scale=0.2]{Oppikirjamaraton-Bitcoin.png} \\
bitcoin:148pMeTViRMFBqMVZRnMZ6Hwccq9WTubq1?label=Oppikirjamaraton

\subsubsection*{Lisätietoja kirjasta}

Kirjan virallinen kotisivu: \url{http://avoinoppikirja.fi} \\
Oppikirjamaraton Facebookissa: \url{http://facebook.com/oppikirjamaraton} \\
Oppikirjamaraton IRCnetissä: \#oppikirjamaraton
