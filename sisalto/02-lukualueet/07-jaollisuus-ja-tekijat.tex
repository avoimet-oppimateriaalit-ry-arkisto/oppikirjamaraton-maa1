\section{Jaollisuus ja tekijöihinjako}

\laatikko{
Kokonaisluku $a$ on jaollinen kokonaisluvulla $b$, jos on olemassa kokonaisluku $c$
niin, että $a = b \cdot c$. Tällöin sanotaan myös, että $b$ on $a$:n tekijä.
}

\begin{esimerkki}
\begin{enumerate}[a)]
\item Luku $-12$ on jaollinen luvulla $3$:lla, sillä $-12 = 3 \cdot (-4)$.
\item $-12$ ei ole jaollinen $5$:llä, sillä ei ole kokonaislukua, joka kerrottuna viidellä olisi $12$.
\end{enumerate}
\end{esimerkki}

Yllä jaollisuus määritellään kertolaskun avulla. Jaollisuuden voi määritellä
myös jakolaskun avulla niin, että $a$ on jaollinen $b$:llä, mikäli $a:b$ on
kokonaisluku. Tämä määritelmä vaatii kuitenkin, että $b \neq 0$, joten
sitä ei voida pitää yleispätevänä määritelmänä jaollisuudelle. Se on
kuitenkin monesti yksinkertaisempi tapa ajatella: esimerkiksi $12$ on
jaollinen $3$:lla, koska $12:3 = 4$, joka on kokonaisluku.

\missingfigure{Kuva, jossa on suorakaide, joka on jaettu 3x4 osaan.}

Kaikki luvut ovat jaollisia itsellään ja luvulla $1$. Esimerkiksi $7=7 \cdot 1=1 \cdot 7$, joten $7$ on jaollinen $1$:llä ja $37$:llä.

\laatikko{
Ykköstä suurempaa kokonaislukua sanotaan alkuluvuksi, jos se on jaollinen
ainoastaan luvulla $1$ ja itsellään.
}

Esimerkiksi luvut 2, 3, 5, 7, 11, 13, 17 ja 19 ovat alkulukuja. 

\laatikko{
Aritmetiikan peruslause

Jokainen ykköstä suurempi kokonaisluku voidaan esittää yksikäsitteisesti alkulukujen tulona.
}

Esimerkiksi luku $84$ voidaan kirjoittaa muodossa $2\cdot 2\cdot 3\cdot 7$. Kokeilemalla havaitaan, että 2, 3, ja 7 ovat kaikki alkulukuja. Aritmetiikan peruslauseen nojalla tiedetään, että tämä on ainoa tapa kirjoittaa $84$ alkulukujen tulona - mahdollista kertolaskujärjestyksen vaihtoa lukuunottamatta. Kun luku $84$ esitetään muodossa $2\cdot 2\cdot 3\cdot 7$ on tapana sanoa, että se on \emph{jaettu alkutekijöihin}. Alkutekjät esitetään yleensä kasvavassa numerojärjestyksessä. Jos sama luku esiintyy tekijöissä useampaan kertaan, on se yleensä yleensä tapana merkitä potenssina. Tällöin luku $84$ voitaisiin kirjoittaa tekijöihin jaettuna $2^2\cdot 3\cdot 7$ ja luku $96$ muodossa $2\cdot 2\cdot 2\cdot 2\cdot 2\cdot 3=2^5\cdot 3$.

Luvun alkutekijät voi löytää etsimällä luvulle ensin jonkin esityksen kahden luvun tulona. Näiden kahden luvun ei tarvitse olla alkulukuja. Sen jälkeen sama toistetaan näille kahdelle luvulle ja edelleen aina uusille luvuille, kunnes tulossa on jäljellä vain alkulukuja. Esimerkiksi luvun $96$ alkutekijät voi löytää vaikkapa seuraavanlaisella ketjulla: $96 = 2 \cdot 48 = 2 \cdot (2 \cdot 24) = 2 \cdot 2 \cdot (6 \cdot 4) = 2 \cdot 2 \cdot (2 \cdot 3) \cdot (2 \cdot 2)$. Nyt jäljellä on vain alkulukuja ja saatu tulo voidaan kirjoittaa lyhennettynä $96 = 2^5 \cdot 3$.

\begin{tehtava}
Mitkä seuraavista luvuista ovat jaollisia luvulla $4$? Jos luku $a$ on jaollinen luvulla $4$, kerro, millä kokonaisluvulla $b$ pätee $a = 4 \cdot b$.\\
a) 1 \quad b) 12  \quad c) 13 \quad d) 2 \quad e) -20 \quad f) 0

\begin{vastaus}
\begin{enumerate}[a)]
	\item Ei ole jaollinen luvulla 4
	\item On jaollinen luvulla 4, $12 = 4 \cdot 3$
	\item Ei ole jaollinen luvulla 4
	\item Ei ole jaollinen luvulla 4
	\item On jaollinen luvulla 4, $-20 = 4 \cdot (-5)$
	\item On jaollinen luvulla 4, $0 = 4 \cdot 0$
\end{enumerate}
\end{vastaus}
\end{tehtava}

\begin{tehtava}
Mitkä seuraavista luvuista ovat alkulukuja? Jos luku ei ole alkuluku, esitä se joidenkin kahden kokonaisluvun (jotka eivät ole ykkönen ja luku itse) tulona.\\
a) 6 \quad b) 11 \quad c) 29 \quad d) -27 \quad e) -11 \quad f) 0

\begin{vastaus}
\begin{enumerate}[a)]
	\item Ei ole alkuluku, esim. $6 = 2 \cdot 3$
	\item On alkuluku
	\item On alkuluku
	\item Ei ole alkuluku, esim. $27 = 3 \cdot (-9)$
	\item Ei ole alkuluku, esim. $-11 = (-1) \cdot 11$ Huom. alkuluvut ovat suurempia kuin yksi (ja siis positiivisia)
	\item Ei ole alkuluku, esim. $0 = 6 \cdot 0$
\end{enumerate}
\end{vastaus}
\end{tehtava}

\begin{tehtava}
Jaa seuraavat luvut alkutekijöihin.\\
a) 12 \quad b) 15 \quad c) 28 \quad d) 30 \quad e) 64 \quad f) 90 \quad g) 100

\begin{vastaus}
\begin{enumerate}[a)]
	\item $12 = 2^2 \cdot 3$
	\item $15 = 3 \cdot 5$
	\item $28 = 2^2 \cdot 7$
	\item $30 = 2 \cdot 3 \cdot 5$
	\item $64 = 2^6$
	\item $90 = 2 \cdot 3^2 \cdot 5$
	\item $100 = 2^2 \cdot 5^2$
\end{enumerate}
\end{vastaus}
\end{tehtava}