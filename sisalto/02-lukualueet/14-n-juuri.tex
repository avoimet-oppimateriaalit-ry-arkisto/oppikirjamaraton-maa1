
\section{Korkeampia juuria}


Yhtälön $x^n=a$ ratkaisujen avulla voidaan määritellä $n$:s juuri mille tahansa positiiviselle kokonaisluvulle $n$. Neliä- ja kuutiojuurten tapauksesta voidaan kuitenkin voi huomata, että kuutiojuuri on määritelty kaikille luvuille, mutta neliöjuuri vain ei-negatiivisille luvuille. Tämä toistuu myös muissa juurissa: parilliset ja parittomat juuret on määriteltävä erikseen.

Juurimerkinnällä $\sqrt[n]{a}$ (luetaan \emph{$n$:s juuri} luvusta $a$) tarkoitetaan lukua, joka toteuttaa ehdon $(\sqrt[n]{a})^n = a$. Jotta juuri olisi yksikäsitteisesti määritelty asetetaan lisäksi, että parillisessa tapauksessa $n$:s juuri tarkoittaa kyseisen yhtälön epänegatiivista ratkaisua.
%($\sqrt{a}, \sqrt[4]{a}, \sqrt[6]{a}$\ldots) vaadittava, että $b\ge0$.

%\subsection{parilliset juuret}

\laatikko{Luvun $a$ $n$.s juuri (luetaan \emph{$n$:s juuri}) on epänegatiivinen luku, jonka neliö on $a$. Tämä voidaan ilmaista myös $(\sqrt[n]{a})^n=a$.}


\laatikko{
{\bf $n$:s juuri}

Luvun $a$ $n$:s juuri on epänegatiivinen luku, jonka $n$:s potenssi on $a$. Tämä voidaan ilmaista myös $(\sqrt[n]{a})^n=a$.}

Luvun toista juurta, eli neliöjuurta $\sqrt[2]{a}$ merkitään myös $\sqrt{a}$.

\laatikko{
Luvun luvun $a$ parillinen juuri on $n$:s juuri, kun $n$ on parillinen luku:
$\sqrt[n]{a}^n=a$, $a\ge0$.
}

Kuten kuutiojuuren tapauksessa, parittomat juuret määritellään kaikille reaaliluvuille $a$.

\laatikko{
Luvun luvun $a$ pariton juuri on $n$:s juuri, kun $n$ on pariton luku:
 $\sqrt[n]{a}^n$, kaikilla $a\in \mathrm{R}$.
}

Korkeampien juurten laskeminen tapahtuu tavallisesti laskimella.

\begin{esimerkki}
Laske.
\begin{enumerate}[a)]
\item $\sqrt[4]{256}$
\item $\sqrt[5]{-243}$
\item $\sqrt[4]{-8}$
\end{enumerate}

{\bf Ratkaisut.}

a) Laskimella saadaan $\sqrt[4]{256}=4$ koska $4^2=256$.

b) Laskimella $\sqrt[4]{-243}=-3$, koska $(-3)^5=-243$.

c) Luvun $-8$ neljäs juuri $\sqrt[4]{-8}$ ei ole määritelty, koska minkään luvun neljäs potenssi ei ole negatiivinen.

{\bf Vastaukset.}

a) 4, b) -3, c) ei määritelty.

\end{esimerkki}

%Mitä näille kahdelle seuraavalle tehdään?
%$\sqrt[n]{ab}=\sqrt[n]{a}\sqrt[n]{b}$
%Jos n on parillinen, niin on lisäksi vaadittava, että $a\ge0$ ja $b\ge0$.
%
%$\sqrt[n]{\frac{a}{b}}=\frac{\sqrt[n]{a}}{\sqrt[n]{b}}$
%Jos n on parillinen, niin on lisäksi vaadittava, että $a\ge0$ ja $b\ge0$.

