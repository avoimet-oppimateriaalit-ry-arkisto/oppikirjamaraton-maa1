
\section{Korkeampia juuria}


Yhtälön $x^n=a$ ratkaisujen avulla voidaan määritellä $n$:s juuri mille tahansa positiiviselle kokonaisluvulle $n$. Neliä- ja kuutiojuurten tapauksesta voidaan kuitenkin voi huomata, että kuutiojuuri on määritelty kaikille luvuille, mutta neliöjuuri vain ei-negatiivisille luvuille. Tämä toistuu myös muissa juurissa: parilliset ja parittomat juuret on määriteltävä erikseen.

Juurimerkinnällä $\sqrt[n]{a}$ (luetaan \emph{$n$:s juuri} luvusta $a$) tarkoitetaan lukua, joka toteuttaa ehdon $(\sqrt[n]{a})^n = a$. Jotta juuri olisi yksikäsitteisesti määritelty asetetaan lisäksi, että parillisessa tapauksessa $n$:s juuri tarkoittaa kyseisen yhtälön epänegatiivista ratkaisua.
%($\sqrt{a}, \sqrt[4]{a}, \sqrt[6]{a}$\ldots) vaadittava, että $b\ge0$.

%\subsection{parilliset juuret}

\laatikko{Luvun $a$ $n$.s juuri (luetaan \emph{$n$:s juuri}) on epänegatiivinen luku, jonka neliö on $a$. Tämä voidaan ilmaista myös $(\sqrt[n]{a})^n=a$.}


\laatikko{
{\bf $n$:s juuri}

Luvun $a$ $n$:s juuri on epänegatiivinen luku, jonka $n$:s potenssi on $a$. Tämä voidaan ilmaista myös $(\sqrt[n]{a})^n=a$.}

Luvun toista juurta, eli neliöjuurta $\sqrt[2]{a}$ merkitään myös $\sqrt{a}$.

\laatikko{
Luvun luvun $a$ parillinen juuri on $n$:s juuri, kun $n$ on parillinen luku:
$(\sqrt[n]{a})^n=a$, $a\ge0$.
}

Kuten kuutiojuuren tapauksessa, parittomat juuret määritellään kaikille reaaliluvuille $a$.

\laatikko{
Luvun luvun $a$ pariton juuri on $n$:s juuri, kun $n$ on pariton luku:
 $(\sqrt[n]{a})^n$, kaikilla $a\in \mathrm{R}$.
}

Korkeampien juurten laskeminen tapahtuu tavallisesti laskimella.

\begin{esimerkki}
Laske.
\begin{enumerate}[a)]
\item $\sqrt[4]{256}$
\item $\sqrt[5]{-243}$
\item $\sqrt[4]{-8}$
\end{enumerate}

{\bf Ratkaisut.}

a) Laskimella saadaan $\sqrt[4]{256}=4$ koska $4^2=256$.

b) Laskimella $\sqrt[4]{-243}=-3$, koska $(-3)^5=-243$.

c) Luvun $-8$ neljäs juuri $\sqrt[4]{-8}$ ei ole määritelty, koska minkään luvun neljäs potenssi ei ole negatiivinen.

\laatikko{Luvun $a$ $n$:s juuri (luetaan \emph{ännäs juuri}) on ei-negatiivinen luku, jonka neliö on $a$. Tämä voidaan ilmaista lyhyemmin $\sqrt[n]{b^n}=b$.}


\begin{tehtava}
Laske.
a) $\sqrt{64}$ \quad b) $\sqrt{-64}$ \quad c) $\sqrt[3]{64}$ \quad d) $\sqrt[3]{-64}$

\begin{vastaus}
a) 8 b) Ei määritelty c) 4 d) -4
\end{vastaus}
\end{tehtava}

\begin{tehtava}
Laske.
a) $\sqrt[4]{81}$ \quad b) $\sqrt[4]{-81}$ \quad c) $\sqrt[5]{32}$ \quad d) $\sqrt[5]{-32}$

\begin{vastaus}
a) 3 b) Ei määritelty c) 2 d) -2
\end{vastaus}
\end{tehtava}

\begin{tehtava}
Laske luvun $10$ potensseja: $10^1, 10^2, 10^3, 10^4, \ldots$ Kuinka monta nollaa on luvussa $10^n$? Laske sitten $\sqrt[6]{1~000~000}$ ja $\sqrt[10]{10~000~000~000}$.

\begin{vastaus}
$10^1 = 10, 10^2 = 100, 10^3 = 1~000, 10^4 = 10~000$. Luvussa $10^n$ on $n$ kappaletta nollia. Niinpä $\sqrt[6]{1~000~000} = 10$ ja $\sqrt[10]{10~000~000~000} = 10$.
\end{vastaus}
\end{tehtava}

\begin{tehtava}
Onko annettu juuri määritelty kaikilla luvuilla $a$? Millaisia arvoja juuri voi saada luvusta $a$ riippuen?\\
a) $\sqrt[4]{a^2}$ \quad b) $\sqrt[4]{-a^2}$ \quad c) $\sqrt[4]{(-a)^2}$ \quad d) $- \sqrt[4]{a^2}$

\begin{vastaus}

\begin{enumerate}[a)]
	\item Juuri on määritelty kaikilla luvuilla $a$, koska kaikkien lukujen neliöt ovat vähintään nolla. Vastaus on aina ei-negatiivinen.
	\item Juuri on määritelty vain luvulla $a = 0$. Muilla $a$:n arvoilla $-a^2$ on negatiivinen, jolloin parillinen juuri ei ole määritelty. Ainoa vastaus, joka voidaan saada, on siis $\sqrt[4]{0} = 0$.
	\item Juuri on määritelty kaikilla luvuilla $a$, koska $(-a)^2$ on aina vähintään nolla. Vastaus on aina ei-negatiivinen.
	\item Juuri on määritelty kaikilla luvuilla $a$, koska kaikkien lukujen neliöt ovat vähintään nolla. Vastaus on aina ei-positiivinen, koska $\sqrt[4]{a^2}$ on aina ei-negatiivinen.
\end{enumerate}
\end{vastaus}
\end{tehtava}

\begin{tehtava}
Onko annettu juuri määritelty kaikilla luvuilla $a$? Millaisia arvoja juuri voi saada luvusta $a$ riippuen?\\
a) $\sqrt[5]{a^2}$ \quad b) $\sqrt[5]{a^3}$ \quad c) $\sqrt[5]{-a^2}$ \quad d) $- \sqrt[5]{a^2}$

\begin{vastaus}

\begin{enumerate}[a)]
	\item Juuri on määritelty kaikilla luvuilla $a$. Vastaus on aina ei-negatiivinen.
	\item Juuri on määritelty kaikilla luvuilla $a$. Vastaus voi olla mikä tahansa luku.
	\item Juuri on määritelty kaikilla luvuilla $a$. Vastaus on aina ei-positiivinen.
	\item Juuri on määritelty kaikilla luvuilla $a$. Vastaus on aina ei-positiivinen.
\end{enumerate}
\end{vastaus}
\end{tehtava}
