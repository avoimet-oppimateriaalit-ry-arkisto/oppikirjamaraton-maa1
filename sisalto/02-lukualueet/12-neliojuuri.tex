\section{Neliöjuuri}

\laatikko{Luvun $a$ neliöjuuri on ei-negatiivinen luku, jonka neliö on $a$. Tämä voidaan ilmaista lyhyemmin $\sqrt{b^2}=b$.}

Neliöjuuren määritteleminen $\sqrt{a}^2=a$ ei johda samaan lopputulokseen. Pohdi, miksi näin on.
%%vai parempi antaa suoraan $\sqrt{a}^2=a$, kun $a \ge 0$
Jatkossa tälaisia määritelmän pieniä muokkauksia ja niistä aiheutuvia muutoksia olisi aina hyvä pohdiskella -- saattavat jopa auttaa muistamaan määritelmän oikean muodon.
%%%%%%%%%%%%%%% ONKO ITSEISARVO KÄSITELTY!!!!! %%%%%%%%%%%%%%%%%%%%%%%%%

%Määritelmäksi ei kelpaisi tämäkään. $\sqrt{a^2}=|a|$ EI OLE KÄSITELTY. Tulee esimerkkinä funktiosta funktioaiheen jälkeen.

Neliöjuurta ei siis nyt määritelty ollenkaan negatiivisille luvuille.

%yhtälöt tulevat vasta myöhemin, siksi esimerkit köyhiä

Esimerkki
\begin{align*}
\sqrt{4} = 2 \quad \textrm{, koska $2>0$ ja $2^2 =4$} 
\end{align*}

%pythagoraan lause on pitänyt käydä ennen tätä!!
Taulutelevision kooksi on ilmoitettu mainoksessa $46''$ ja kuvasuhteeksi 19:6. Kuinka leveä televisio on arviolta?
Vastaus: $44''$ tai 111 cm

