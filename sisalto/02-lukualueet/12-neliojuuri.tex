\section{Neliöjuuri}

Ajatellaan, että neliön pinta-ala on $a$. Halutaan tietää, mikä on kyseisen neliön sivun pituus. Vastausta tähän kysymykseen kutsutaan luvun $a\ge 0$ neliöjuureksi ja merkitään $\sqrt{a}$. Luvun $a$ neliöjuuri on myös yhtälön $x^2 = a$ vastaus. Tällöin täytyy kuitenkin huomata, että myös luku $x=-\sqrt{a}$ toteutaa kyseisen yhtälön. Neliöjuurella tarkoitetaan kyseisen yhtälön epänegatiivista vastausta. Tämä on luonnollista, koska neliön sivun pituus ei voi olla negatiivinen luku.

\laatikko{Luvun $a$ neliöjuuri on epänegatiivinen luku, jonka neliö on $a$. Tämä voidaan ilmaista lyhyemmin $\sqrt{a}^2=a$.}

Neliöjuurta ei tällä kurssilla määritellä negatiivisille luvuille, koska neliön pinta-ala on aina positiivinen luku tai nolla. Käytännössä lukujen neliöjuuria lasketaan laskimella.


\begin{esimerkki}
Laske.
\begin{enumerate}[a)]
\item $\sqrt{4}$

\item $\sqrt{144}$

\item $\sqrt{3471}$.
\end{enumerate}

{\bf Ratkaisut.}

a)
Laskimella tai päässä laskemalla nähdään, että $\sqrt{4} = 2$, koska $2>0$ ja $2^2 =4$.

b) 
Laskimella saadaan $\sqrt{144}=12$. Tämä voidaan vielä tarkistaa laskemalla $12^2 = 12\cdot 12=144$.

c)
Laskimella saadaan $\sqrt{4471}\approx 58,9$.

{\bf Vastaukset.}
a) $2$, b) $12$, c) $58,9$.

\end{esimerkki}

% Huom! Vaatii Pythagoraan lauseen.
Taulutelevision kooksi (halkaisijaksi) on ilmoitettu mainoksessa $46,0$" \, ($116,8$ cm) ja kuvasuhteeksi 16:9. Kuinka leveä televisio on arviolta?
Vastaus: $40,7$" \, ($103,4$ cm).

