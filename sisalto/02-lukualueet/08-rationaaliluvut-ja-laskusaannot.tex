\chapter{Rationaaliluvut ja laskusäännöt}

\emph{Rationaaliluvulla} tarkoitetaan lukua $q$, joka voidaan esittää muodossa
\[
q=\frac{a}{b}, 
\]
missä $a$ ja $b$ ovat kokonaislukuja ja $b\neq 0$. Rationaalilukujen joukkoa merkitään symbolilla $\mathbb{Q}$. Tällaista esitystä kutsutaan \emph{murtoluvuksi}. Tässä esityksessä lukua $a$ kutsutaan \emph{osoittajaksi} ja lukua $n$ \emph{nimittäjäksi}. Kaikki rationaaliluvut voidaan esittää murtolukuina, mutta tämä ei ole ainoa tapa rationaalilukujen esittämiseksi. 

\laatikko{
Murtoluku on muotoa
\[
\frac{a}{b}
\]
missä $a,b\in \mathbb{Z}$ ja $a\neq 0$. Jokainen murtoluku on rationaaliluku ja jokainen rationaaliluku voidaan esittää murtolukuna.
}

Seuraavaksi tutkitaan laskutoimituksia murtoluvuilla. Muotoa $a/c$ ja $b/c$ olevien murtolukujen yhteenlaskussa lukujen osoittajat lasketaan yhteen:
\[
\frac{a}{c} + \frac{b}{c} = \frac{a+b}{c}.
\]
Tällaisia murtolukuja joilla on sama nimittäjä kutsutaan samannimisiksi. Jos lasketaan yhteen murtolukuja joilla on eri nimittäjät, ne täytyy kertoa samannimisiksi eli \emph{laventaa} ennen kuin ne voidaan laskea yhteen. Jos siis $a/b$ ja $c/d$ ovat murtolukuja ja $b\neq d$, lasketaan
\[
\frac{a}{b} + \frac{c}{d} = \frac{ad}{bd} + \frac{bc}{bd} = \frac{ad+bc}{bd}
\]
Tässä luku $a/b$ lavennetaan luvun $c/d$ nimittäjällä $d$ ja vastaavasti luku $c/d$ lavennetaan ensimmäisen yhteelaskettavan nimittäjällä $b$. Näin saadaan kaksi samannimistä murtolukua, joiden kummankin nimittäjä on yhteenlaskettavien nimittäjien tulo $bd$.

\begin{esimerkki}
Laske
\[
\frac{1}{2} + \frac{1}{6} + \frac{2}{6}.
\]

{\bf Ratkaisu.}
Aluksi lavennetaan luvut saman nimisiksi ja sen jälkeen lasketaan ne yhteen:
\begin{eqnarray*}
\frac{1}{2} + \frac{1}{6} + \frac{2}{6}
&=&
\frac{3\cdot 1}{\cdot 2} + \frac{1}{6} + \frac{2}{6}\\
&=&
\frac{3}{6} + \frac{1}{6} + \frac{2}{6}\\
&=& \frac{3+1+2}{6}\\ &=& \frac{6}{6} = 1.
\end{eqnarray*}

{\bf Vastaus.} $1$.

\end{esimerkki}

Murtolukujen $a/b$ ja $c/d$ tulo lasketaan kertomalla lukujen osoittajat ja nimittäjät keskenään:
\[
\frac{a}{b}\cdot \frac{c}{d} = \frac{a\cdot c}{b\cdot d} = \frac{ac}{cd}.
\]
Rationaaliluvun $q\neq 0$ \emph{käänteisluvulla} tarkoitetaan sellaista lukua $q^{-1}$, jolle pätee
\[
q\cdot q^{-1} = 1.
\]
Jos rationaaliluku $q\neq 0$ on esitetty murtolukumuodossa $q=a/b$, niin sen käänteisluku saadaan vaihtamalla osoittaja ja nimittäjä: $q^{-1} = b/a$.

Murtolukujen $p=a/b$ ja $q=c/d\neq 0$ \emph{osamäärä} $p/q$ saadaan, kun kerrotaan luku $p$ luvun $q$ käänteisluvulla:
\[
\frac{p}{q} = p\cdot q^{-1} = \frac{a}{b}\cdot\Big(\frac{c}{d}\Big)^{-1} = \frac{a}{b}\cdot \frac{d}{c}
= \frac{ad}{bc}.
\]



\laatikko{
{\bf Murtolukujen laskusääntöjä}

Yhteenlasku
\begin{equation}
\frac{a}{b} + \frac{c}{d} = \frac{ad}{bd} + \frac{bc}{bd} = \frac{ad+bc}{bd}
\end{equation}
Kertolasku
\begin{equation}
\frac{a}{b}\cdot \frac{c}{d} = \frac{a\cdot c}{b\cdot d} = \frac{ac}{cd}.
\end{equation}
Jakolasku
\begin{equation}
\frac{a}{b}\Big/\frac{c}{d}= \frac{a}{b}\cdot \frac{d}{c}
= \frac{ad}{bc}.
\end{equation}
}

Kahta murtolukua vertailtaessa kannattaa ensin laventaa ne saman nimisiksi.

\begin{esimerkki}
Salamipizza jaetaan kuuteen ja tonnikalapizza neljään yhtä suureen siivuun. Vesa saa kaksi siivua salamipizzaa ja yhden siivun tonnikalapizzaa. Minttu saa kaksi siivua tonnikalapizzaa. Kumpi saa enemmän pizzaa, jos molemmat pizzat ovat saman kokoisia?

{\bf Ratkaisu.}

Huomataan, että $12 = 3\cdot 4 = 2\cdot 6$. Siten ratkaisussa esiintyvät luvut kannattaa laventaa niin, että nimittäjänä on luku $12$. Vesan saama määrä pizzaa on
\[
\frac{2}{6} + \frac{1}{4} = \frac{2\cdot 2}{2\cdot 6} + \frac{3\cdot 1}{3\cdot 4} 
=\frac{4}{12}+\frac{3}{12} = \frac{7}{12}.
\]
Mintun saama määrä pizzaa on
\[
\frac{2}{4} = \frac{3\cdot 2}{3\cdot 4} = \frac{6}{12}.
\]
Koska $6/12 < 7/12$, Vesa saa enemmän.

{\bf Vastaus.} Vesa saa enemmän.
\end{esimerkki}


%tähän kuva

\laatikko{
Kokonaisluvun voi esittää murtolukuna asettamalle sen nimittäjäksi luvun yksi.
\begin{equation}
2 + \frac{1}{3} = \frac{2}{1} + \frac{1}{3} = \frac{3 \cdot 2}{3 \cdot 1} + \frac{1}{3} = \frac{6+1}{3} = \frac{7}{3}
\end{equation}
}

Kaksi ja yksi kolmasosa karkkipussillista karkkia on sama määrä pahoinvointia kuin seitsemän kolmasosakarkkipussillista karkkia.

%enemmän asiaa prosenteista

Yksi prosentti vastaa yhtä sadasosaa: $1 \% = \frac{1}{100}$

Laske %aika randomit luvut
\begin{tehtava}
    a) $\frac{6}{2} + \frac{3}{5}$
    b) $\frac{7}{8} - \frac{1}{4}$
    c) $2 \frac{1}{3} + \frac{4}{6}$
    c) $4 \frac{7}{2} - 6 \frac{5}{4}$
    
    \begin{vastaus}
        a) $\frac{18}{5}$
        b) $\frac{5}{8}$
        c) $3$
        d) $-\frac{41}{6}$
    \end{vastaus}
\end{tehtava}

\begin{tehtava}
    a) $\frac{1}{3} \cdot \frac{6}{5}$
    b) $\frac{5}{4} \cdot (-\frac{2}{3})$
    c) $\frac{2}{5} (2 - \frac{3}{4})$
    c) $(\frac{5}{6} - \frac{1}{3})(\frac{7}{4} - \frac{3}{2})$
    
    \begin{vastaus}
        a) $\frac{2}{5}$
        b) $\frac{5}{6}$
        c) $\frac{1}{2}$
        d) $\frac{1}{8}$
    \end{vastaus}
\end{tehtava}

\begin{tehtava} %lisää kakkaa
    a) $ \frac{\frac{3}{7} + \frac{5}{4}}{3}$
    b) $ \frac{\frac{10}{8}}{\frac{5}{2}}$
    c) $ \frac{\frac{1}{3} - \frac{5}{10}}{\frac{3}{4} + \frac{1}{2}}$
    d) $ 3\frac{\frac{4}{2} + \frac{10}{4}}{\frac{3}{2} - \frac{2}{3}}$
    
    \begin{vastaus}
        a) $\frac{47}{28}$
        b) $\frac{1}{2}$
        c) $-\frac{1}{3}$
        d) $\frac{54}{5}$
    \end{vastaus}
\end{tehtava}

\begin{tehtava} %lisää kakkaa
    Pontus, Viljami, Jarkko-Kaaleppi, Ahmed ja Milla leipoivat lanttuvompattipiirakkaa.
    Pontus kuitenkin söi piirakasta kolmanneksen ennen muita, ja loput piirakasta
    jaetaan muiden kanssa tasan. Kuinka suuren osan muut saavat?
    
    \begin{vastaus}
        Muut saavat piirakasta kuudesosan.
    \end{vastaus}
\end{tehtava}

\begin{tehtava} %ja lisää
    Huvipuiston sisäänpääsylippu maksaa 20 euroa, ja lapset pääsevät puoleen
    hintaan. Avajaispäivänä sisään pääsee 25\% halvemmalla. Kuinka paljon kolmen
    lapsen yksinhuoltajaperheelle maksaa päästä sisään avajaispäivänä?
    
    \begin{vastaus}
        37,50 euroa
    \end{vastaus}
\end{tehtava}
