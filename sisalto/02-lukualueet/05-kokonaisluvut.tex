\chapter{Kokonaisluvut}

Yksinkertaisimmat käyttämämme luvut ovat lukumäärien ilmaisemiseen käyttämämme luvut $0, 1, 2, 3, \ldots$. Kutsumme niitä \emph{luonnollisiksi luvuiksi} ja merkitsemme niiden joukkoa symbolilla $\mathbb{N}$. Joissain yhteyksissä nollan ei katsota kuuluvan luonnollisiin lukuihin. 

Luonnollisille luvuille $m$ ja $n$ on määritelty tavallinen yhteenlasku $m + n$. Luonnollisten lukujen $m$ ja $n$ kertolasku määritellään peräkkäisinä yhteenlaskuina
\[m \cdot n = \underbrace{m + m + \ldots + m}_{n\text{ kpl}} = \underbrace{n + n + \ldots + n}_{m\text{ kpl}}.\]
Nollalla kertomisen ajatellaan olevan "tyhjä yhteenlasku"\ eli nolla.

Luonnollisten lukujen $m$ ja $n$ erotus $m-n$ on luku $k$, jolle $k + n = m$. Havaitsemme kuitenkin, että esimerkiksi erotus $0-1$ ei ole luonnollinen luku. Ratkaisemme ongelman määrittelemällä kullekin luonnolliselle luvulle $n$ vastaluvun $-n$, jolle $n + (-n) = 0$. Näin saamme kokonaislukujen joukon
\[\mathbb{Z} = \{\ldots, -2, -1, 0, 1, 2, \ldots\}.\]
Kokonaisluvuissa kaikilla $m$ ja $n$ voidaan määritellä $m-n = m+(-n)$, missä $+$ merkitsee jälleen tavallista yhteenlaskua.

