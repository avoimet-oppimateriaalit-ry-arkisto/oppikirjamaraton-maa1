\section{Kuutiojuuri}

Kuution tilavuus on $a$. Halutaan tietää, mikä on kyseisen kuution sivun pituus. Vastausta tähän kysymykseen kutsutaan luvun $a\ge 0$ kuutiojuureksi ja merkitään $\sqrt[3]{a}$. Luvun $a$ kuutiojuuri on myös yhtälön $x^3 = a$ vastaus.

\laatikko{Luvun $a$ kuutiojuuri on luku, jonka kuutio $a^3$ on $a$. Tämä voidaan ilmaista myös $(\sqrt[3]{a})^3=a$.}

Jos $a<0$, niin kysymys kuutiosta jonka tilavuus on $a$, ei ole mielekäs. Tästä huolimatta kuutiojuuri määritellään myös negatiivisille luvuille. Syy tähän on, että yhtälöllä $x^3=a$ on tässäkin tapauksessa yksikäsitteinen ratkaisu. Positiivisen luvun kuutiojuuri on aina positiivinen ja negatiivisen luvun kuutiojuuri on aina negatiivinen luku. Kuutiojuuren voi siis ottaa mistä tahansa reaaliluvusta.


\begin{esimerkki}
Laske.
\begin{enumerate}[a)]
\item $\sqrt[3]{27}$

\item $\sqrt[3]{1397}$

\item $\sqrt[3]{2197}$.
\end{enumerate}

{\bf Ratkaisut.}

a)
Laskimella tai päässä laskemalla nähdään, että $\sqrt[3]{27} = 3$, koska  $3^3 =3\cdot 3\cdot 3=27$.

b) 
Laskimella saadaan $\sqrt[3]{1397}\approx 11,18$. 

c)
Laskimella saadaan $\sqrt[3]{2197}=13$.
Tämä voidaan vielä tarkistaa laskemalla $13^3 = 13\cdot 13\cdot 13=2197$.

{\bf Vastaukset.}

a) $3$, b) $11,18$, c) $13$.
\end{esimerkki}

