\chapter{Numerot ja luvut}



\laatikko{Länsimaisessa traditiossa käytössämme on kymmenen numeromerkkiä: 0, 1, 2, 3, 4, 5, 6, 7, 8 ja 9. Näitä kutsutaan alkuperänsä mukaan hindu-arabialaisiksi numeroiksi.}

Yksittäisellä numeromerkillä ei kuitenkaan ole vielä matematiikassa tarvittavaa lukuarvoa, vaan luvut rakennetaan yhdistelemällä numeroita.

\begin{esimerkki}
Luku \[715531\] koostuu numeroista 7, 1, 5, 5, 3 ja 1.
\end{esimerkki}



Äidinkielellinen huomautus...




\begin{esimerkki}
III=1+1+1=3
IX=10-1=9
XII=10+1+1=12
XIX=10-1+10=19
CDX=500-100+10=410
MCMD=1 000+1 000-100+500




\section{Jos tämä on matematiikkaa, miksi käytämme kirjaimia?}

suuruus, yhtäsuuruus, eri suuret

symbolit, muuttujat, akiot, kreikkalaisia...

Erilaisilla luvuilla voidaan suorittaa erilaisia laskutoimituksia. Seuraavissa luvuissa esitellään ja käydään läpi lukiomatematiikassa ja mahdollisissa jatko-opinnoissa käytettäviä lukujoukkoja ja tavallisimmat laskutoimitukset.

\section{Paikka- ja lukujärjestelmät}

Luvut $134$ ja $413$ eivät ole sama luku; saamme eri lukuja, kun numeroita yhdistellään eri tavoin.

