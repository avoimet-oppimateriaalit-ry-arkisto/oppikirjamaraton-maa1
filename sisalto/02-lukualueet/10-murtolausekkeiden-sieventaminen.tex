\chapter{Murtolausekkeiden sieventäminen}

\laatikko{
Jos nimittäjässä on eri luku, murtoluvut pitää ensin kertoa samannimisiksi eli \emph{laventaa}, jotta ne voi laskea yhteen.
\begin{equation}
\frac{a}{b} + \frac{c}{d} = \frac{ad}{bd} + \frac{bc}{bd} = \frac{ad+bc}{bd}
\end{equation}
}

Kumpi lapsi saa enemmän pizzaa: tyttö, joka saa kaksi kolmasosasiivua ($ \frac{2}{3}$) vai poika, joka saa kolme neljäsosasiivua ($ \frac{3}{4}$)? Huomataan että $4*3=12$. Jos ajatellaankin kummankin siivuja kahdestatoistaosina, osuuksia on helpompi vertailla. Tyttö saa kahdeksan kahdestoistaosaa, koska $ \frac{2}{3} = \frac{2 \cdot 4}{3 \cdot 4} = \frac{8}{12}$. Poika saa yhdeksän kahdestatoistaosaa, koska $ \frac{3}{4} = \frac{3 \cdot 3}{3 \cdot 4} = \frac{9}{12}$. Poika saa siis enemmän.

%tähän kuva

\laatikko{
Kokonaisluvun voi esittää murtolukuna asettamalle sen nimittäjäksi luvun yksi.
\begin{equation}
2 + \frac{1}{3} = \frac{2}{1} + \frac{1}{3} = \frac{3 \cdot 2}{3 \cdot 1} + \frac{1}{3} = \frac{6+1}{3} = \frac{7}{3}
\end{equation}
}

\laatikko{
Jos murtoluvun osoittajassa tai nimittäjässä on summa, jonka osilla on yhteinen tekijä, sen voi ottaa \emph{yhteiseksi tekijäksi} sulkujen eteen. Jos osoittajassa ja nimittäjässä on sen jälkeen sama kerroin, sen voi jakaa pois molemmista eli \emph{supistaa} pois.
\begin{equation}
\frac{ac+bc}{c} = \frac{ \cancel{c} (a+b)}{\cancel{c}} = a+b
\end{equation}
}

\laatikko{
Joskus murtolauseke sieventyy, jos sen esittääkin kahden murtoluvun summana.
\begin{equation}
\frac{ca+b}{c} = \frac{ca}{c} + \frac{b}{c} = a + \frac{b}{c}
\end{equation}
}

\laatikko{
Samantyyppiset asiat voidaan laskea yhteen tai \emph{ryhmitellä}.
\begin{equation}
ax^2 + bx + cx^2 + dy + ex = (a+b)x^2 + (b+c)x + dy
\end{equation}
}

\begin{esimerkki}

$ \frac{1}{6} + \frac{3}{2} = \frac{1}{2\cdot 3} + \frac{3}{2} = \frac{1}{2 \cdot 3} + \frac{3 \cdot 3}{2 \cdot 3} = \frac{1}{6} + \frac{9}{6} = \frac{10}{6} = \frac{\cancel{2} \cdot 5}{\cancel{2} \cdot 3} = \frac{5}{3}$

\end{esimerkki}

\begin{tehtava}
Sievennä
\begin{enumerate}
\item $\frac{2x^3}{x}$
\item $\frac{6x^2+8y}{2x^2}$
\item $ \frac{1-x}{3} + \frac{x+2}{6}$
\item $ \frac{5x-1}{3} - \frac{2x+5}{2}$
\end{enumerate}
\begin{vastaus}
\begin{enumerate}
\item $2x^2$
\item $3+4y$
\item $ -\frac{x}{6}$
\item $ \frac{2}{3} x - \frac{17}{6}$
\end{enumerate}
\end{vastaus}
\end{tehtava}

