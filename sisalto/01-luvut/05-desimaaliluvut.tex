\chapter{Desimaaliluvut}

\emph{Desimaaliluvut} on \emph{kymmenjärjestelmään} perustava tapa merkitä rationaalilukuja. Niillä voi merkitä myös irrationaalilukujen \emph{likiarvoja}.

$123,456$ on esimerkki desimaaliluvusta.

\begin{itemize}
	\item $123$ on sen \emph{kokonaisosa}.
	\item Kokonaisluku erotetaan loppuosasta \emph{desimaalierottimella}, joka on suomen kielessä pilkku (,).
	\footnote{Useissa englanninkielisissä maissa käytetään desimaalierottimena pistettä ja pilkku on tuhaterotin. Esimerkiksi yhdysvaltalaisessa kirjassa merkintä \$ 1,000.50 tarkoittaa tuhatta dollaria ja 50 senttiä. Suomen kielessä tuhaterotin on välilyönti.}
	\item Osaa $456$ kutsutaan desimaaliluvun \emph{loppuosaksi}.
\end{itemize}

\laatikko{

Esimerkkinä annettu desimaaliluku tulkitaan seuraavasti:
\begin{equation}
123,456 = 1 \cdot 10^2 + 2 \cdot 10^1 + 3 \cdot 10^0 + 4 \cdot 10^{-1} + 5 \cdot 10^{-2} + 6 \cdot 10^{-3}
\end{equation}

\missingfigure{Tähän kuva siitä, kuinka desimaaliluvun numerot vastaavat kymmenen potensseja}

}

Kymmenjärjestelmä saa nimensä siitä, että jokainen luvussa esiintyvä numero kertoo sen paikkaa vastaavien kymmenen potenssien määrän.


Desimaaliluvut voidaan muuttaa murtoluvuiksi laskemalla ne auki ylläolevan tavan mukaan.

\begin{esimerkki}
$21,37 = 2 \cdot 10^1 + 1 \cdot 10^0 + 3 \cdot 10^{-1} + 7 \cdot 10^{-2} = 2 \cdot 10 + 1 \cdot 1 + 3 \cdot \frac{1}{10} + 7 \cdot \frac{1}{100} = 21 + \frac{3}{10} + \frac{7}{100} = \frac{21 \cdot 100}{100} + \frac{3 \cdot 10}{10 \cdot 10} + \frac{7}{100} = \frac{21 \cdot 100 + 3 \cdot 10 + 7}{100} = \frac{2100+30+7}{100} = \frac{2137}{100}$
\end{esimerkki}

Toisaalta päästään paljon helpommalla, kun huomataan, että voidaan vain kertoa ja jakaa koko luku ''niin monella kympillä, kuin on numeroita desimaalipilkun jälkeen''. Täsmällisemmin sanottuna $10^n$:llä, jossa $n$ on pilkun jälkeen tulevien numeroiden määrä.

\begin{esimerkki}
$21,47 = 21,47 \cdot 1 = 21,47 \cdot \frac{10^2}{10^2} = \frac{21,47 \cdot 100}{100} = \frac{2147}{100}$
\end{esimerkki}

\begin{esimerkki}
$0,007 = 0,007 \cdot 1 = 0,007 \cdot \frac{10^3}{10^3} = \frac{0,007 \cdot 1000}{1000} = \frac{7}{1000}$
\end{esimerkki}

Menetelmän toimivuus kaikkien desimaalilukujen tapauksessa voidaan todistaa tarkastelemalla desimaaliluvun määritelmää, mutta jätetään tässä tekemättä.

\laatikko{
Tässä on joitain murtolukujen desimaaliesityksiä, jotka on hyvä osata
\begin{itemize}
	\item $ \frac{1}{10} = 0,1$
	\item $ \frac{1}{100} = 0,01$
	\item $ \frac{1}{2} = 0,5$
	\item $ \frac{1}{4} = 0,25$
	\item $ \frac{3}{4} = 0,75$
\end{itemize}
}

\section*{Tehtäviä}

\begin{tehtava}
Laske ja totea edellä olevien murtolukujen desimaaliesitysten paikkansapitävyys.
\end{tehtava}

\begin{tehtava}
Muuta murtoluvuksi.
%selkeys voitti täsmällisyyden ei siis murtolukumuotoon
	\begin{enumerate}[a)]
		\item $43{,}532$
		\item $5{,}031$
		\item $0{,}23$
		\item $0{,}3002$
		\item $0{,}101$
	\end{enumerate}
\begin{vastaus}
	\begin{enumerate}[a)]
		\item $ \frac{43532}{1000}$
		\item $ \frac{5031}{1000}$
		\item $ \frac{23}{100}$
		\item $ \frac{3002}{1000}$
		\item $ \frac{101}{1000}$
	\end{enumerate}
\end{vastaus}
\end{tehtava}

\begin{tehtava}%sovteht tai vaikea tehtävä? sivevennyksen takia
Muuta murtoluvuksi ja sievennä.
%selkeys voitti täsmällisyyden ei siis murtolukumuotoon
	\begin{enumerate}[a)]
		\item $0{,}01$
		\item $0{,}0245$
		\item $0{,}004$
		\item $0{,}001004$
	\end{enumerate}
\begin{vastaus}
	\begin{enumerate}[a)]
		\item $ \frac{1}{100}$
		\item $ \frac{49}{200}$
		\item $ \frac{1}{250}$
		\item $ \frac{251}{250\ 000}$
	\end{enumerate}
\end{vastaus}
\end{tehtava}

\todo{rat.lukujen muuttaminen desimaaliluvuiksi, motivaatio, teoria ja esimerkkejä}
\todo{tehtäviä rat.lukujen muuttamisesta desimaaliluvuiksi (huom! ei päättymättömiä desimaaliluvuista)}

\begin{tehtava}
	\begin{enumerate}[a)]
		\item
	\end{enumerate}
\begin{vastaus}
	\begin{enumerate}[a)]
		\item
	\end{enumerate}
\end{vastaus}
\end{tehtava}

\begin{tehtava}
	\begin{enumerate}[a)]
		\item
	\end{enumerate}
\begin{vastaus}
	\begin{enumerate}[a)]
		\item
	\end{enumerate}
\end{vastaus}
\end{tehtava}

\laatikko{
Joissain yhteyksissä käytetään muitakin \emph{lukujärjestelmiä} kuin kymmenjärjestelmää. Esimerkiksi tietokoneet käyttävät kaksikantajärjestelmää eli \emph{binäärijärjestelmää}. Binäärijärjestelmässä jokainen luvun numero kertoo sen paikkaa vastaavan kakkosen potenssien määrän kymmenen potenssien sijasta.

Binäärijärjestelmässä esitetty luku merkitään yleensä kirjoittamalla pieni kakkonen luvun jälkeen, esim. $10,01_2$.
}

\begin{esimerkki}
$10,01_2 = 1 \cdot 2^1 + 0 \cdot 2^0 + 0 \cdot 2^{-1} + 1 \cdot 2^{-2} = 2,25_{10}$
\end{esimerkki}

\section*{Tehtäviä}

\begin{tehtava}
Muunna seuraavat binääriluvut kymmenjärjestelmään.
	\begin{enumerate}[a)]
		\item $101,0_2$
		\item $1,00101_2$
		\item $100101,1101_2$
	\end{enumerate}
\begin{vastaus}
	\begin{enumerate}[a)]
		\item $5,0_{10}$
		\item $1,15625_{10}$
		\item $37.8125_{10}$
	\end{enumerate}
\end{vastaus}
\end{tehtava}

\begin{tehtava}
Muunna seuraavat luvut binäärijärjestelmään.
	\begin{enumerate}[a)]
		\item $7,0_{10}$
		\item $2,5_{10}$
		\item $11,1875_{10}$
	\end{enumerate}
\begin{vastaus}
	\begin{enumerate}[a)]
		\item $111,0_2$
		\item $10,1_2$
		\item $1011,0011_2$
	\end{enumerate}
\end{vastaus}
\end{tehtava}

\todo{päättymättömät desimaaliluvut, jotka on esitettävissä rationaalilukuina}
\todo{nämä on hyvä osata: esimerkkejä desimaaliluvuista jotka on esitettävissä rationaalilukuina, esim. 0,333...}
\todo{tehtäviä päättymättömistä mutta jaksollisista desimaaliluvuista}
