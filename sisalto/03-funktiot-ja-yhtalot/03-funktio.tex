\chapter{Funktio}
Matematiikassa tutkitaan paljon suureiden välisiä riippuvuuksia. Tällaiset riippuvuudet voidaan muotoilla funktioiden avulla. Esimerkiksi tuotteen arvonlisäveroprosentti riippuu tuotteen tyypistä. Tämä riippuvuus voidaan kirjoittaa funktiona eri tuotetyyppien joukolta $A$ reaalilukujen joukolle $\mathbb{R}$, missä funktio liittää jokaiseen tuotteeseen sen arvonlisäveroprosentin.

[Esimerkki, kuva arvonlisäverofunktiosta, missä \[A = \{\text{ahvenfilee}, \text{AIV-rehu}, \text{auto}, \text{runokirja}, \text{ravintola-ateria}, \text{särkylääke}, \text{televisio}\},\]$f(\text{ahvenfilee}) = 13$, $f(\text{AIV-rehu}) = 13$, $f(\text{auto}) = 23$, $f(\text{runokirja}) = 9$, $f(\text{ravintola-ateria}) = 13$, $f(\text{särkylääke}) = 9$, $f(\text{televisio}) = 23$]

\laatikko{Funktio $f$ joukosta $A$ joukkoon $B$ on sääntö, joka liittää $A$:n jokaiseen alkioon $x$ täsmälleen yhden $B$:n alkion, jota merkitään $f(x)$. $A$:ta kutsutaan funktion $f$ määrittelyjoukoksi ja $B$:tä funktion $f$ maalijoukoksi.}

Käytäntöjä:
\begin{itemize}
\item $f(x) = y$ lausutaan: "Funktio saa arvon y pisteessä x."
\item Funktioita voidaan kutsua myös kuvauksiksi.
\item Funktion määrittely- ja maalijoukot jätetään usein merkitsemättä, jos ne ovat selviä asiayhteydestä. Tällä kurssilla maalijoukko on yleensä reaaliluvut.
\end{itemize}

Funktion arvojoukko koostuu niistä maalijoukon alkioista, jotka funktio saa arvokseen jossakin määrittelyjoukon pisteessä.

Funktio määritellään usein kaavan avulla. Voimme esimerkiksi määritellä funktion $f$ reaaliluvuilta reaaliluvuille kaavalla
\[f(x) = x^2 + 1,\]
eli $f$ liittää jokaiseen reaalilukuun $x$ luvun $x^2+1$. Tästä funktiosta huomaamme, että arvojoukko ei ole aina sama kuin maalijoukko. Esimerkiksi luku $0$ kuuluu funktion $f$ maalijoukkoon, mutta ei funktion $f$ arvojoukkoon, koska $x^2+1\geq 1$ neliön epänegatiivisuuden perusteella. [kuva]

\esimerkki{Määritellään funktio $f$ kaavalla [kuva]
\[f(x) = \frac{1}{x-1}.\]
Funktiolle ei ole erikseen annettu määrittelyjoukkoa, joten se täytyy päätellä asiayhteydestä. Huomataan, että sääntö ei määrittele funktion $f$ arvoa kun $x = 1$, mutta kylläkin kaikilla $x\in\mathbb{R}\setminus\{1\}$ [vai?]. Näin ollen luonnollinen valinta määrittelyjoukoksi on $\mathbb{R}\setminus\{1\}$.
}
