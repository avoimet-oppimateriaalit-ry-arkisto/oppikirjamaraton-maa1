%DOKUMENTTITYYPPI
\documentclass[a4paper,12pt]{memoir}
%KÄYTETTÄVÄT PAKETIT
\usepackage{amssymb,amsmath}		%ams
\usepackage[finnish,english]{babel}			%suomenkielinen tavutus
%\usepackage[T1]{fontenc}			%skanditavutus
\usepackage[utf8]{inputenc}		%skandien syöttö (windows)

%RIVIVÄLI
\linespread{1.3}                      %riviväli 1.3



%--------------------------------
\usepackage{pgf,tikz}
\newenvironment{esimerkki}[1]{Esimerkki.}{Moi}
\usepackage{answers}
\Newassociation{vastaus}{Vastaus}{ans}
\newtheorem{tehtava}{Tehtävä}
\usepackage{config/oppikirjamaraton}
\definecolor{olivine}{RGB}{154,185,115}
\usepackage{mdframed}
\newcommand{\laatikko}[1]{\begin{mdframed}[backgroundcolor=olivine] #1 \end{mdframed}}
\usepackage{cancel}
%\theoremstyle{definition}
\newtheorem{theorem}{Teoreema}
\usepackage{eurosym}
%\usepackage{fancyhdr}
%\pagestyle{fancy}
\renewcommand*{\printchaptername}{Luku}
\renewcommand{\contentsname}{Sisällys}
\usepackage{todonotes}
%--------------------------------

%--------------------------------

%\chapterstyle{bianchi}

%--------------------------------
% CHAPTER-TYYLI (vaihtoehtoinen yo. bianchin kanssa)

\usepackage{fourier} % or what ever
\usepackage[scaled=.92]{helvet}%. Sans serif - Helvetica
\usepackage{color,calc}
\newsavebox{\ChpNumBox}
\definecolor{ChapBlue}{rgb}{1.00,0.70,0}

\makeatletter

\newcommand*{\thickhrulefill}{%
    \leavevmode\leaders\hrule height 1\p@ \hfill \kern \z@
}

\newcommand*\BuildChpNum[2]{%
    \begin{tabular}[t]{
        @{}c@{}
    }
    \makebox[0pt][c]{#1\strut} \\[.5ex]
    \colorbox{ChapBlue}{
        \rule[-10em]{0pt}{0pt}%
        \rule{1ex}{0pt}\color{black}#2\strut
        \rule{1ex}{0pt}
    }
    \end{tabular}
}

\makechapterstyle{YellowBox}{
    \renewcommand{\chapnamefont}{\large\scshape}
    \renewcommand{\chapnumfont}{\Huge\bfseries}
    \renewcommand{\chaptitlefont}{\raggedright\Huge\bfseries}
    \setlength{\beforechapskip}{20pt}
    \setlength{\midchapskip}{26pt}
    \setlength{\afterchapskip}{40pt}
    \renewcommand{\printchaptername}{}
    \renewcommand{\chapternamenum}{}
    \renewcommand{\printchapternum}{
        \sbox{\ChpNumBox}{
            \BuildChpNum{\chapnamefont\@chapapp}{
                \chapnumfont\thechapter
            }
        }
    }
    \renewcommand{\printchapternonum}{
        \sbox{\ChpNumBox}{
            \BuildChpNum{\chapnamefont\vphantom{\@chapapp}}{
                \chapnumfont\hphantom{\thechapter}
            }
        }
    }
    \renewcommand{\afterchapternum}{}
    \renewcommand{\printchaptertitle}[1]{
        \usebox{\ChpNumBox}\hfill
        \parbox[t]{\hsize-\wd\ChpNumBox-1em}{
            \vspace{\midchapskip}
            \thickhrulefill\par
            \chaptitlefont ##1\par
        }
    }
}
\chapterstyle{YellowBox}

%--------------------------------

\begin{document}

\tableofcontents




%\part{Alku}
\chapter{Esipuhe}

%%%%%%%%%%%%%%%%%%%%%%%%%%%%%%%%%%%%%%%%%%%%%%%%%%%%%%%%%%%%%%%%%%%%%%%%%%%%%%%%
%%%%  /usr/share/doc/texlive-fonts-extra-doc/fonts/arev/mathtesty.tex

% mathtesty.tex, by Stephen Hartke 20050522
% based on mathtestx.tex in the mathptmx package
% and symbols.tex by David Carlisle

Matematiikka tarjoaa työkaluja asioiden jäsentämiseen, päättelyyn ja mallintamiseen. Alasta riippuen käsittelemme matematiikassa erilaisia \textbf{objekteja}: Geometriassa tarkastelemme tasokuvioita ja kolmiulotteisia rakenteita. Algebra tutkii lukujen ja funktioiden ominaisuuksia. Todennäköisyyslaskenta arvioi erilaisten tapausten ja tilanteiden mahdollisuuksia ja riskejä. Matemaattinen analyysi (kurssit 7,8 ja 10) tutkii funktioita ja niiden muuttumista.

Jokaiseen tarkastelukohteeseen liitetään myös niille ominaisia \textbf{operaatioita}. Tämä kurssi käsittelee lähinnä lukuja ja niiden operaatioita, joita \textbf{laskutoimituksiksi} kutsutaan. Kirjan ensimmäisessä osassa käsittelemme luvun käsitteen, yleisimmät lukutyypit ja lukujen tavallisimman laskutoimitukset.

\section*{Sananen kirjasta}

Tulimme, kirjoitimme, voitimme.

\section*{Tekijöiden kommentit}

\todo{Miten olisi lyhyt kommentti tai lainaus jokaiselta tekijältä? Jotain yleviä mietteitä kirjasta, rohkaisevia tai nasevia kommentteja lukijalle, alku- tai loppukevennyksiä tai jotain randomia}

\begin{tabular}{cc} 
	\begin{tabular}{c}
	 \textbf{Lauri Hellsten}
	\\ 
	kommentti1 \end{tabular}
&
	\begin{tabular}{c}
	 \textbf{Niko Ilomäki}
	\\ 
	kommentti2 \end{tabular}
\\
	\begin{tabular}{c}
	 \textbf{Tero Keinänen}
	\\ 
	kommentti3 \end{tabular}
&
	\begin{tabular}{c}
	 \textbf{Vesa Linja-aho}
	\\ 
	kommentti4 \end{tabular}
\\
	\begin{tabular}{c}
	 \textbf{Ossi Mauno}
	\\ 
	kommentti1 \end{tabular}
&
	\begin{tabular}{c}
	 \textbf{Joonas Mäkinen}
	\\ 
	kommentti2 \end{tabular}
\\
	\begin{tabular}{c}
	 \textbf{Matti Pajunen}
	\\ 
	kommentti3 \end{tabular}
&
	\begin{tabular}{c}
	 \textbf{Pekka Peura}
	\\ 
	kommentti4 \end{tabular}
\\
	\begin{tabular}{c}
	 \textbf{Annika Piiroinen}
	\\ 
	kommentti1 \end{tabular}
&
	\begin{tabular}{c}
	 \textbf{Kaisa Pohjonen}
	\\ 
	kommentti2 \end{tabular}
\\
	\begin{tabular}{c}
	 \textbf{Antti Rasila}
	\\ 
	kommentti3 \end{tabular}
&
	\begin{tabular}{c}
	 \textbf{Johanna Rämö}
	\\ 
	kommentti4 \end{tabular}		
\\
	\begin{tabular}{c}
	 \textbf{Annika Piiroinen}
	\\ 
	kommentti1 \end{tabular}
&
	\begin{tabular}{c}
	 \textbf{Kaisa Pohjonen}
	\\ 
	kommentti2 \end{tabular}
\\
	\begin{tabular}{c}
	 \textbf{Antti Rasila}
	\\ 
	kommentti3 \end{tabular}
&
	\begin{tabular}{c}
	 \textbf{Johanna Rämö}
	\\ 
	kommentti4 \end{tabular}
	\\
	\begin{tabular}{c}
	 \textbf{Juha Sointu}
	\\ 
	kommentti1 \end{tabular}
&
	\begin{tabular}{c}
	 \textbf{Tommi Sottinen}
	\\ 
	kommentti2 \end{tabular}
\\
	\begin{tabular}{c}
	 \textbf{Jarno Talponen}
	\\ 
	kommentti3 \end{tabular}
&
	\begin{tabular}{c}
	 \textbf{Topi Talvitie}
	\\ 
	kommentti4 \end{tabular}
\\
	\begin{tabular}{c}
	 \textbf{Sampo Tiensuu}
	\\ 
	kommentti3 \end{tabular}
&
	\begin{tabular}{c}
	 \textbf{Ville Tilvis}
	\\ 
	kommentti4 \end{tabular}
	
	  
\end{tabular} 

\section*{Kiitämme}
\begin{itemize}
\item Metropolia
\item TEK
\item Senja Larsen
\item Kebab Pizza Service
\end{itemize}

%%%%%%%%%%%%%%%%%%%%%%%%%%%%%%%%%%%%%%%%%%%%%%%%%%%%%%%%%%%%%%%%%%%%%%%%%%%%%%%%
%%%% /usr/share/doc/texlive-doc-en/fonts/free-math-font-survey/source/textfragment.tex







%%% Local Variables: 
%%% mode: latex
%%% End: 

\chapter{Lähtötasotesti}

(Tää tulee oikeasti ennen tätä chapteria ja osaa)

\begin{tehtava}
\begin{enumerate}
\item Laske $2^2+2 \cdot 2+2$
\item sasdas
\item 
\end{enumerate}

\begin{vastaus}
\begin{enumerate}
\item 
\item
\item

\end{enumerate}
\end{vastaus}
\end{tehtava}


\part{Luvut ja laskutoimitukset}
\chapter{Numerot ja luvut}

Ihmisellä ja muilla eläimillä on luonnostaan matemaattisia taitoja. Monet niistä, esimerkiksi lukumäärien laskeminen, ovat yllättävän monimutkaisia kognitiivisia prosesseja, jotka kehittyvät lapsuudessa – toisilla aiemmin, toisilla myöhemmin. Kaikki koulussa opeteltava peruslaskento ja myös matematiikka tieteen alana rakentavat tämän biologisen osaamisen päälle. Laskeminen itsessään on vain yksi matematiikan osa-alue, eikä kaikki matematiikka ole laskemista.

\sivulaatikko{Huom.! laskea (lukumäärä) englanti count ruotsi \_ /n
laskea (laskutoimitus) englanti calculate , ruotsi \_}

Hyvin olennaisena kehitysaskeleena niin yksilön matemaattiselle ajattelulle kuin yhteiskunnallekin on ollut luonnollisen kielen tavoin kyky merkitä lukumäärien laskemista ja muuta matemaattista pohdintaa kirjalliseen muotoon. On olemassa hyvin monia erilaisia tapoja merkitä lukumääriä. Helpoin tapa ja yksinkertaisin tapa on käyttää vain yhtä samaa merkkiä ja toistaa sitä:

\missingfigure{Piirrettynä "tukkimiehen kirjanpitoa" ja  vertaus arabialaisilla numeroilla}

Käytettävissä olevien merkkien määrää voidaan lisätä, jolloin suuria lukuja voidaan kirjoittaa lyhyemmin. Tämä vastaa myös luonnollisten kielten tilannetta: Suomen kielen aakkosiin kuuluu 29 kirjainta, joista sanat muodostetaan. Sanat voivat olla kuinka pitkiä vain kahdesta kirjaimesta ylöspäin. Mandariinikiinassa sen sijaan käytetään omaa piirrosmerkkiä jokaiselle sanalle. Merkkejä täytyy osata 29 sijaan tuhansia, mutta jokaisen sanan voi kirjoittaa lyhyesti.

Matematiikassa erilaisista numeromerkeistä tai yksinkertaisesti numeroista muodostetaan lukuja yhdistelemällä niitä sopivasti erilaisten paikkajärjestelmien mukaan. Esimerkiksi antiikin Roomassa käytössä olivat numeromerkit I, V, X, L, C , D ja M. Niiden numeroiden vastaavuudet meidän käyttämiimme lukuarvoihin ovat seuraavat:
I=1
V=5
X=10
L=50
C=100
D=500
M=1 000
Huomaa, että suuri osa roomalaisista numeromerkeistä ovat jo itsessään arvoltaan niin suuria, että me tarvitsemme niiden nykyilmaisuun monta merkkiä! Nollaa roomalaisissa numeroissa ei ole, ja tiettävästi tuhatta suurempia arvoja esittäviä numeromerkkejä merkkejä otettiin käyttöön vasta keskiajalla. 
Lukuja koostetaan näistä merkeistä siten, että merkit kirjoitetaan peräkkäin pääasiassa laskevassa järjestyksessä ja niiden numeroarvot lasketaan yhteen. Jos arvoltaan pienempi numeromerkki (korkeintaan yksi) edeltää suurempaa, pienempi vähennetään suuremmasta ennen yhteenlaskun jatkamista. 

Luvut $134$ ja $413$ eivät ole sama luku; saamme eri lukuja, kun numeroita yhdistellään eri tavoin.

\begin{esimerkki}
III=1+1+1=3
IX=10-1=9
XII=10+1+1=12
XIX=10-1+10=19
CDX=500-100+10=410
MCMD=1 000+1 000-100+500
\end{esimerkki}


\laatikko{Länsimaisessa traditiossa käytössämme on kymmenen numeromerkkiä: 0, 1, 2, 3, 4, 5, 6, 7, 8 ja 9. Näitä kutsutaan alkuperänsä mukaan hindu-arabialaisiksi numeroiksi.}

Yksittäisellä numeromerkillä ei kuitenkaan ole vielä matematiikassa tarvittavaa lukuarvoa, vaan luvut rakennetaan yhdistelemällä numeroita.

\begin{esimerkki}
Luku \[715531\] koostuu numeroista 7, 1, 5, 5, 3 ja 1.
\end{esimerkki}



\sivulaatikko{Englannin kielen sana \textit{number} voi viitata sekä numeroon että lukuun. Sana \textit{digit} tarkoittaa pelkästään yhtä numeromerkkiä. Ruotsiksi luku on \textit{tal}, lukumäärä \textit{antal} ja numeroa tai lukumäärää tarkoittamatonta numeroyhdistelmää kuvaa suomen kielen tapaan sana \textit{nummer}.}



postinumero, puhelinnumero!


\begin{esimerkki}
Selitys kymmenjärjestelmästä, kymmenet, sadat, tuhannet, ...
\[20661,43\]
\end{esimerkki}

\sivulaatikko{Kielioppihuomautuksia: 1) Tuhaterottimena käytetään välilyöntiä, ei pilkkua tai pistettä. 2) Suomessa on käytössä desimaalipilkku, ei -piste! Yhdysvaltalaiset ovat suunnitelleet laskimesi.}


\section{Jos tämä on matematiikkaa, miksi käytämme kirjaimia?}

Numeroita kuvaavat merkit ovat mielivaltaisia symboleita. Lukujakin edustamaan päädytään joskus käyttämään jotakin lyhennysmerkintää

yleisyys

suuruus, yhtäsuuruusmerkintä, eri suuret

$x=2$
-> $2=x$
symbolit, muuttujat, alkiot, kreikkalaisia...

\sivulaatikko{Painotekstissä kirjaimella merkityt muuttujat kirjoitetaan \textit{kursiivilla} ja aina saman arvon saavat matemaattiset vakiot kuten $\pi$ pystyyn.}

\section{Paikka- lukujärjestelmät}





Erilaisilla luvuilla voidaan suorittaa erilaisia laskutoimituksia. Seuraavissa luvuissa esitellään ja käydään läpi lukiomatematiikassa ja mahdollisissa jatko-opinnoissa käytettäviä lukujoukkoja ja tavallisimmat laskutoimitukset.

\chapter{Luonnolliset luvut}

(Joonas jatkaa tästä vielä!)

Suomen kielen verbi 'laskea' voi tarkoittaa matematiikassa kahta eri asiaa: lukumäärien laskemista ja laskutoimitusten suorittamista.

\laatikko{laskea (lukumäärä) englanti count ruotsi \_ /n
laskea (laskutoimitus) englanti calculate , ruotsi \_}

Ihmisellä ja muilla eläimillä on luonnostaan matemaattisia taitoja. Monet niistä, esimerkiksi lukumäärien laskeminen, ovat yllättävän monimutkaisia kognitiivisia prosesseja, jotka kehittyvät lapsuudessa – toisilla aiemmin, toisilla myöhemmin. Kaikki koulussa opeteltava peruslaskento ja myös matematiikka tieteen alana rakentavat tämän biologisen osaamisen päälle. Laskeminen itsessään on vain yksi matematiikan osa-alue, eikä kaikki matematiikka ole laskemista. Huomaa, että suomen kielen verbillä laskea tarkoitetaan sekä lukumäärien laskemista (engl. counting) että lukujen laskutoimitusten suorittamista (engl. calculating).

Hyvin olennaisena kehitysaskeleena niin yksilön matemaattiselle ajattelulle kuin yhteiskunnallekin on ollut luonnollisen kielen tavoin kyky merkitä lukumäärien laskemista ja muuta matemaattista pohdintaa kirjalliseen muotoon.  On olemassa hyvin monia erilaisia tapoja merkitä lukumääriä. Helpoin tapa ja yksinkertaisin tapa on käyttää vain yhtä samaa merkkiä ja toistaa sitä. Jos

käytettävissä olevien merkintöjä määrää voidaan lisätä, jolloin suuria lukuja voidaan kirjoittaa lyhyemmin. Tämä vastaa myös luonnollisten kielten tilannetta: Suomen kielen aakkosiin kuuluu 29 kirjainta, joista sanat muodostetaan. Sanat voivat olla kuinka pitkiä vain kahdesta kirjaimesta ylöspäin. Kiinassa sen sijaan käytetään omaa piirrosmerkkiä jokaiselle sanalle. Merkkejä täytyy osata 29 sijaan tuhansia, mutta jokaisen sanan voi kirjoittaa lyhyesti. 
Matematiikassa erilaisista numeromerkeistä tai yksinkertaisesti numeroista muodostetaan lukuja yhdistelemällä niitä sopivasti erilaisten paikkajärjestelmien mukaan. Esimerkiksi antiikin Roomassa käytössä olivat numeromerkit I, V, X, L, C , D ja M. Niiden numeroiden vastaavuudet meidän käyttämiimme lukuarvoihin ovat seuraavat:
I=1
V=5
X=10
L=50
C=100
D=500
M=1 000
Huomaa, että suuri osa roomalaisista numeromerkeistä ovat jo itsessään arvoltaan niin suuria, että me tarvitsemme niiden nykyilmaisuun monta merkkiä! Nollaa roomalaisissa numeroissa ei ole, ja tiettävästi tuhatta suurempia arvoja esittäviä numeromerkkejä merkkejä otettiin käyttöön vasta keskiajalla. 
Lukuja koostetaan näistä merkeistä siten, että merkit kirjoitetaan peräkkäin pääasiassa laskevassa järjestyksessä ja niiden numeroarvot lasketaan yhteen. Jos arvoltaan pienempi numeromerkki (korkeintaan yksi) edeltää suurempaa, pienempi vähennetään suuremmasta ennen yhteenlaskun jatkamista. 

%\chapter{Joukko-oppia}

(voisi integroida lukuoppiin) T: Joonas

%\chapter{Logiikkaa}

(voisi integroida yhtälöiden teoriaan, sinne saa hyvin ekvivalenssin, implikaation ja disjunktion ja nepä ovat ne, mitä juuri tarvitaan) T: Joonas

% Luonnolliset luvut esitellään nyt kokonaislukujen alussa.

% \sivulaatikko{engl. \emph{natural numbers, counting numbers} ruots. \emph{naturliga tal}}
% 
% 
% \laatikko{Luonnollisia lukuja käytetään kolmeen eri tarkoitukseen:
% 
% \begin{enumerate}
% \item Lukumäärien ilmoittamiseen (kardinaaliluvut)
% \item Järjestyksen ilmoittamiseen (ordinaaliluvut)
% \item Indeksointiin ja asioiden nimeämiseen
% \end{enumerate}
% }


% \section{Tehtäviä}
% 
% \begin{tehtava}
% 
% Onko kardinaali vai ordinaali vai indeksointi?
% 
% \end{tehtava}

\chapter{Kokonaisluvut}

Yksinkertaisimmat käyttämämme luvut ovat lukumäärien ilmaisemiseen käytetyt $0,
1, 2, 3, \ldots$. Näitä kutsutaan \emph{luonnollisiksi luvuiksi}, ja niiden
joukkoa eli kaikkia luonnollisia lukuja yhdessä merkitään symbolilla
$\mathbb{N}$. Edellä nolla määriteltiin luonnolliseksi luvuksi, mutta tästä
ei ole yhteistä sopimusta: jotkut pitävät nollaa luonnollisena lukuna ja
toiset eivät.

Luonnollisille luvuille $m$ ja $n$ on määritelty yhteenlasku $m + n$, esimerkiksi
$5 + 3 = 8$.
Luonnollisten lukujen $m$ ja $n$ kertolasku määritellään peräkkäisinä yhteenlaskuina
\[m \cdot n = \underbrace{m + m + \ldots + m}_{n\text{ kpl}} = \underbrace{n + n + \ldots + n}_{m\text{ kpl}}.\]
Nollalla kertomisen ajatellaan olevan "tyhjä yhteenlasku"\ eli nolla,
$0 \cdot m = 0$.

Luonnollisten lukujen $m$ ja $n$ erotus määritellään yhteenlaskun avulla:
$m-n$ on luku $k$, jolle $k + n = m$. Kahden luonnollisen luvun erotus
ei kuitenkaan aina ole luonnollinen luku, esimerkkinä $3 - 5$.
Ratkaisemme ongelman määrittelemällä kullekin luonnolliselle
luvulle $n$ vastaluvun $-n$, jolle $n + (-n) = 0$.

Luonnolliset luvut ja niiden vastaluvut muodostavat yhdessä
kokonaislukujen joukon
\[\mathbb{Z} = \{\ldots, -2, -1, 0, 1, 2, \ldots\}.\]
Kun käytämme kokonaislukuja, voidaan kahden luvun erotus määritellä
yhteenlaskun ja vastaluvun avulla yksinkertaisesti $m-n = m+(-n)$.
\chapter{Kokonaislukujen aritmetiikkaa}

Tiivistelmä...

Kysymys: Mitä saadaan, kun luvusta $5$ vähennetään luku $-8$?

Negatiivisten ja positiivisten lukujen yhteen- ja vähennyslaskut on helppoa ymmärtää lukusuoran avulla.

Allaolevien esimerkkien yhteyteen kuvat lukusuoralla!

$5+8$ "viiteen lisätään $8$"

$5+(+8)$ "viiteen lisätään $+8$" $+8$ tarkoittaa samaa kuin $8$. '$+$'-merkkiä käytetään luvun edessä silloin, kun halutaan korostaa, että kyseessä on nimenomaan positiivinen luku.

$5-(+8)$ "viidestä vähennetään $+8$" Tämä tarkoittaa samaa kuin 5-8. Lukusuoralla siis liikutaan 8 pykälää taaksepäin.

$5+(-8)$ "viiteen lisätään $-8$" Mitä tapahtuu, kun lisätään negatiivinen luku? Kun lukuun lisätään 1, se kasvaa yhdellä. Kun lukuun lisätään 0, se ei kasva lainkaan. Eikö tällöin ole luonnollista ajatella, että kun lisätään luku, joka on pienempi kuin nolla, täytyisi lopputuloksesta tulla vielää pienempi. Tällä logiikalla negatiivisen luvun lisäämisen pitäisi siis pienentää alkuperäistä lukua. Siksi on sovittu, että $5+(-8)$ on yhtä suuri kuin $5-8$.

5-(-8) "viidestä vähennetään $-8$" Negatiivisen luvun lisääminen on vastakohtainen positiivisen luvun lisäämiselle. Tällöin olisi luonnillista, että negatiivisen luvun vähentäminen olisi myös vastakohtaista positiivisen luvun vähentämiselle. Kun positiivisen luvun vähentäminen pienentää lukua, pitäisi negatiivisen luvun vähentämisen siis kasvattaa lukua. Tämän vuoksi onkin sovittu, että $5-(-8)$ tarkoittaa samaa kuin $5+8$. Usein on myös tapana sanoa, että kaksi miinusmerkkiä kumoavat toisensa, jolloin lopputulos on positiivinen.

Samaan logiikkaan perustuen on sovittu myös merkkisäännöt positiivisten ja negatiivisten lukujen kertolaskuissa. Kun negatiivinen ja positiivinen luku kerrotaan keskenään, saadaan negatiivinen luku, mutta kun kaksi negatiivista lukua kerrotaan keskenään, saadaan positiivinen luku.

Seuraavista kuvat lukusuoralle:

$3*4$ "kolme kappaletta nelosia"

$3*(-4)$ "kolme kappaletta miinus-nelosia"

$-3*4$ "miinus-kolme kappaletta nelosia"

$-3*(-4)$ "miinus-kolme kappaletta miinus-nelosia"


\section{Jaollisuus ja tekijöihinjako}

\laatikko{
Kokonaisluku $a$ on jaollinen kokonaisluvulla $b$, jos on olemassa kokonaisluku $c$
niin, että $a = b \cdot c$. Tällöin sanotaan myös, että $b$ on $a$:n tekijä.
}

\begin{esimerkki}
\begin{enumerate}[a)]
\item Luku $-12$ on jaollinen luvulla $3$:lla, sillä $-12 = 3 \cdot (-4)$.
\item $-12$ ei ole jaollinen $5$:llä, sillä ei ole kokonaislukua, joka kerrottuna viidellä olisi $12$.
\end{enumerate}
\end{esimerkki}

Yllä jaollisuus määritellään kertolaskun avulla. Jaollisuuden voi määritellä
myös jakolaskun avulla niin, että $a$ on jaollinen $b$:llä, mikäli $a:b$ on
kokonaisluku. Tämä määritelmä vaatii kuitenkin, että $b \neq 0$, joten
sitä ei voida pitää yleispätevänä määritelmänä jaollisuudelle. Se on
kuitenkin monesti yksinkertaisempi tapa ajatella: esimerkiksi $12$ on
jaollinen $3$:lla, koska $12:3 = 4$, joka on kokonaisluku.

\missingfigure{Kuva, jossa on suorakaide, joka on jaettu 3x4 osaan.}

Kaikki luvut ovat jaollisia itsellään ja luvulla $1$. Esimerkiksi $7=7 \cdot 1=1 \cdot 7$, joten $7$ on jaollinen $1$:llä ja $37$:llä.

\laatikko{
Ykköstä suurempaa kokonaislukua sanotaan alkuluvuksi, jos se on jaollinen
ainoastaan luvulla $1$ ja itsellään.
}

Esimerkiksi luvut 2, 3, 5, 7, 11, 13, 17 ja 19 ovat alkulukuja. 

\laatikko{
Aritmetiikan peruslause

Jokainen ykköstä suurempi kokonaisluku voidaan esittää yksikäsitteisesti alkulukujen tulona.
}

Esimerkiksi luku $84$ voidaan kirjoittaa muodossa $2\cdot 2\cdot 3\cdot 7$. Kokeilemalla havaitaan, että 2, 3, ja 7 ovat kaikki alkulukuja. Aritmetiikan peruslauseen nojalla tiedetään, että tämä on ainoa tapa kirjoittaa $84$ alkulukujen tulona - mahdollista kertolaskujärjestyksen vaihtoa lukuunottamatta. Kun luku $84$ esitetään muodossa $2\cdot 2\cdot 3\cdot 7$ on tapana sanoa, että se on \emph{jaettu alkutekijöihin}. Alkutekjät esitetään yleensä kasvavassa numerojärjestyksessä. Jos sama luku esiintyy tekijöissä useampaan kertaan, on se yleensä yleensä tapana merkitä potenssina. Tällöin luku $84$ voitaisiin kirjoittaa tekijöihin jaettuna $2^2\cdot 3\cdot 7$ ja luku $96$ muodossa $2\cdot 2\cdot 2\cdot 2\cdot 2\cdot 3=2^5\cdot 3$.

Luvun alkutekijät voi löytää etsimällä luvulle ensin jonkin esityksen kahden luvun tulona. Näiden kahden luvun ei tarvitse olla alkulukuja. Sen jälkeen sama toistetaan näille kahdelle luvulle ja edelleen aina uusille luvuille, kunnes tulossa on jäljellä vain alkulukuja. Esimerkiksi luvun $96$ alkutekijät voi löytää vaikkapa seuraavanlaisella ketjulla: $96 = 2 \cdot 48 = 2 \cdot (2 \cdot 24) = 2 \cdot 2 \cdot (6 \cdot 4) = 2 \cdot 2 \cdot (2 \cdot 3) \cdot (2 \cdot 2)$. Nyt jäljellä on vain alkulukuja ja saatu tulo voidaan kirjoittaa lyhennettynä $96 = 2^5 \cdot 3$.

\begin{tehtava}
Mitkä seuraavista luvuista ovat jaollisia luvulla $4$? Jos luku $a$ on jaollinen luvulla $4$, kerro, millä kokonaisluvulla $b$ pätee $a = 4 \cdot b$.\\
a) 1 \quad b) 12  \quad c) 13 \quad d) 2 \quad e) -20 \quad f) 0

\begin{vastaus}
\begin{enumerate}[a)]
	\item Ei ole jaollinen luvulla 4
	\item On jaollinen luvulla 4, $12 = 4 \cdot 3$
	\item Ei ole jaollinen luvulla 4
	\item Ei ole jaollinen luvulla 4
	\item On jaollinen luvulla 4, $-20 = 4 \cdot (-5)$
	\item On jaollinen luvulla 4, $0 = 4 \cdot 0$
\end{enumerate}
\end{vastaus}
\end{tehtava}

\begin{tehtava}
Mitkä seuraavista luvuista ovat alkulukuja? Jos luku ei ole alkuluku, esitä se joidenkin kahden kokonaisluvun (jotka eivät ole ykkönen ja luku itse) tulona.\\
a) 6 \quad b) 11 \quad c) 29 \quad d) -27 \quad e) -11 \quad f) 0

\begin{vastaus}
\begin{enumerate}[a)]
	\item Ei ole alkuluku, esim. $6 = 2 \cdot 3$
	\item On alkuluku
	\item On alkuluku
	\item Ei ole alkuluku, esim. $27 = 3 \cdot (-9)$
	\item Ei ole alkuluku, esim. $-11 = (-1) \cdot 11$ Huom. alkuluvut ovat suurempia kuin yksi (ja siis positiivisia)
	\item Ei ole alkuluku, esim. $0 = 6 \cdot 0$
\end{enumerate}
\end{vastaus}
\end{tehtava}

\begin{tehtava}
Jaa seuraavat luvut alkutekijöihin.\\
a) 12 \quad b) 15 \quad c) 28 \quad d) 30 \quad e) 64 \quad f) 90 \quad g) 100

\begin{vastaus}
\begin{enumerate}[a)]
	\item $12 = 2^2 \cdot 3$
	\item $15 = 3 \cdot 5$
	\item $28 = 2^2 \cdot 7$
	\item $30 = 2 \cdot 3 \cdot 5$
	\item $64 = 2^6$
	\item $90 = 2 \cdot 3^2 \cdot 5$
	\item $100 = 2^2 \cdot 5^2$
\end{enumerate}
\end{vastaus}
\end{tehtava}
\chapter{Rationaaliluvut ja laskusäännöt}

\laatikko{
Jos nimittäjässä on eri luku, murtoluvut pitää ensin kertoa samannimisiksi eli \emph{laventaa}, jotta ne voi laskea yhteen.
\begin{equation}
\frac{a}{b} + \frac{c}{d} = \frac{ad}{bd} + \frac{bc}{bd} = \frac{ad+bc}{bd}
\end{equation}
}

Kumpi lapsi saa enemmän pizzaa: tyttö, joka saa kaksi kolmasosasiivua ($ \frac{2}{3}$) vai poika, joka saa kolme neljäsosasiivua ($ \frac{3}{4}$)? Huomataan että $4*3=12$. Jos ajatellaankin kummankin siivuja kahdestatoistaosina, osuuksia on helpompi vertailla. Tyttö saa kahdeksan kahdestoistaosaa, koska $ \frac{2}{3} = \frac{2 \cdot 4}{3 \cdot 4} = \frac{8}{12}$. Poika saa yhdeksän kahdestatoistaosaa, koska $ \frac{3}{4} = \frac{3 \cdot 3}{3 \cdot 4} = \frac{9}{12}$. Poika saa siis enemmän.

\missingfigure{tähän kuva pizzoista}

\laatikko{
Kokonaisluvun voi esittää murtolukuna asettamalle sen nimittäjäksi luvun yksi.
\begin{equation}
2 + \frac{1}{3} = \frac{2}{1} + \frac{1}{3} = \frac{3 \cdot 2}{3 \cdot 1} + \frac{1}{3} = \frac{6+1}{3} = \frac{7}{3}
\end{equation}
}

Kaksi ja yksi kolmasosa karkkipussillista karkkia on sama määrä pahoinvointia kuin seitsemän kolmasosakarkkipussillista karkkia.

\todo{enemmän asiaa prosenteista}

Yksi prosentti vastaa yhtä sadasosaa: $1 \% = \frac{1}{100}$

Laske %aika randomit luvut

\begin{tehtava}
a) $\frac{3}{5} + \frac{1}{5}$
b) $\frac{5}{7} + \frac{4}{7}$
c) $2 + \frac{2}{3}$
d) $3 + \frac{3}{5} + \frac{2}{5}$    
    \begin{vastaus}
        a) $\frac{4}{5}$
				b) $\frac{9}{7} = 1 \frac{2}{7}$
				c) $2 \frac{2}{3} = \frac{8}{3}$
				d) $3 \frac{5}{5} = 3+1 = 4$
    \end{vastaus}
\end{tehtava}

\begin{tehtava}
    a) $\frac{6}{2} + \frac{3}{5}$
    b) $\frac{7}{8} - \frac{1}{4}$
    c) $2 \frac{1}{3} + \frac{4}{6}$
    d) $4 \frac{7}{2} - 6 \frac{5}{4}$
    
    \begin{vastaus}
        a) $\frac{18}{5}$
        b) $\frac{5}{8}$
        c) $3$
        d) $-\frac{41}{6}$
    \end{vastaus}
\end{tehtava}

\begin{tehtava}
    a) $2 \cdot \frac{2}{5}$
    b) $2 \cdot \frac{2}{3}$
    c) $\frac{5}{4} \cdot 2 \cdot 3$
    c) $\frac{\frac{3}{7}}{4}$
    
    \begin{vastaus}
		a) $\frac{4}{5}$
		b) $\frac{4}{3} = 1 \frac{1}{3}$
		c) $\frac{15}{2} = 7 \frac{1}{2}$
		d) $\frac{3}{28}$
    \end{vastaus}
\end{tehtava}

\begin{tehtava}
    a) $\frac{1}{3} \cdot \frac{6}{5}$
    b) $\frac{5}{4} \cdot (-\frac{2}{3})$
    c) $\frac{2}{5} (2 - \frac{3}{4})$
    c) $(\frac{5}{6} - \frac{1}{3})(\frac{7}{4} - \frac{3}{2})$
    
    \begin{vastaus}
        a) $\frac{2}{5}$
        b) $-\frac{5}{6}$
        c) $\frac{1}{2}$
        d) $\frac{1}{8}$
    \end{vastaus}
\end{tehtava}

\begin{tehtava} %lisää kakkaa
    a) $ \frac{\frac{3}{7} + \frac{5}{4}}{3}$
    b) $ \frac{\frac{10}{8}}{\frac{5}{2}}$
    c) $ \frac{\frac{1}{3} - \frac{5}{10}}{\frac{3}{4} + \frac{1}{2}}$
    d) $ 3\frac{\frac{4}{2} + \frac{10}{4}}{\frac{3}{2} - \frac{2}{3}}$
    
    \begin{vastaus}
        a) $\frac{47}{28}$
        b) $\frac{1}{2}$
        c) $-\frac{1}{3}$
        d) $\frac{54}{5}$
    \end{vastaus}
\end{tehtava}

\begin{tehtava} %lisää kakkaa
    Pontus, Viljami, Jarkko-Kaaleppi, Ahmed ja Milla leipoivat lanttuvompattipiirakkaa.
    Pontus kuitenkin söi piirakasta kolmanneksen ennen muita, ja loput piirakasta
    jaetaan muiden kanssa tasan. Kuinka suuren osan muut saavat?
    
    \begin{vastaus}
        Muut saavat piirakasta kuudesosan.
    \end{vastaus}
\end{tehtava}

\begin{tehtava} %ja lisää
    Huvipuiston sisäänpääsylippu maksaa 20 euroa, ja lapset pääsevät puoleen
    hintaan. Avajaispäivänä sisään pääsee 25\% halvemmalla. Kuinka paljon kolmen
    lapsen yksinhuoltajaperheelle maksaa päästä sisään avajaispäivänä?
    
    \begin{vastaus}
        37,50 euroa
    \end{vastaus}
\end{tehtava}

\chapter{Potenssisäännöt}

Potenssilla $2^4$ tarkoitetaan tuloa $2\cdot 2\cdot 2\cdot 2$.
\begin{equation}
\text{Eli} 2^4=2\cdot 2\cdot 2\cdot 2=16.
\end{equation}
Lausekkeessa $2^4$ luku 2 on \textbf{kantaluku} ja luku 4 on \textbf{eksponentti}.

\begin{esimerkki}
\textbf{Esimerkki 1}
\begin{equation}
\text{a)} (-2)^3=(-2)\cdot (-2)\cdot (-2)=-8
\end{equation}

\begin{equation}
\text{b)} (-2)^4=(-2)\cdot (-2)\cdot (-2)\cdot (-2)=16
\end{equation}

\begin{equation}
\text{c)} -2^4=-2\cdot 2\cdot 2\cdot 2=-16
\end{equation}

\begin{equation}
\text{d)} 2^2\cdot 2^3=\underbrace{2\cdot 2}_{2 kpl}\cdot \underbrace{2\cdot 2\cdot 2}_{3 kpl}=2^5=32
\end{equation}

\begin{equation}
\text{e)}\frac{2^4\cdot 2^2}{2^3}=\frac{\overbrace{2\cdot 2\cdot 2\cdot \cancel{2}}\cdot \overbrace{\cancel{2}\cdot \cancel{2}}}{\cancel{2}\cdot \cancel{2}\cdot \cancel{2}}
\end{equation}
\end{esimerkki}


\chapter{Murtolausekkeiden sieventäminen}

\laatikko{
Jos nimittäjässä on eri luku, murtoluvut pitää ensin kertoa samannimisiksi eli \emph{laventaa}, jotta ne voi laskea yhteen.
\begin{equation}
\frac{a}{b} + \frac{c}{d} = \frac{ad}{bd} + \frac{bc}{bd} = \frac{ad+bc}{bd}
\end{equation}
}

Kumpi lapsi saa enemmän pizzaa: tyttö, joka saa kaksi kolmasosasiivua ($ \frac{2}{3}$) vai poika, joka saa kolme neljäsosasiivua ($ \frac{3}{4}$)? Huomataan että $4*3=12$. Jos ajatellaankin kummankin siivuja kahdestatoistaosina, osuuksia on helpompi vertailla. Tyttö saa kahdeksan kahdestoistaosaa, koska $ \frac{2}{3} = \frac{2 \cdot 4}{3 \cdot 4} = \frac{8}{12}$. Poika saa yhdeksän kahdestatoistaosaa, koska $ \frac{3}{4} = \frac{3 \cdot 3}{3 \cdot 4} = \frac{9}{12}$. Poika saa siis enemmän.

%tähän kuva

\laatikko{
Kokonaisluvun voi esittää murtolukuna asettamalle sen nimittäjäksi luvun yksi.
\begin{equation}
2 + \frac{1}{3} = \frac{2}{1} + \frac{1}{3} = \frac{3 \cdot 2}{3 \cdot 1} + \frac{1}{3} = \frac{6+1}{3} = \frac{7}{3}
\end{equation}
}

\laatikko{
Jos murtoluvun osoittajassa tai nimittäjässä on summa, jonka osilla on yhteinen tekijä, sen voi ottaa \emph{yhteiseksi tekijäksi} sulkujen eteen. Jos osoittajassa ja nimittäjässä on sen jälkeen sama kerroin, sen voi jakaa pois molemmista eli \emph{supistaa} pois.
\begin{equation}
\frac{ac+bc}{c} = \frac{ \cancel{c} (a+b)}{\cancel{c}} = a+b
\end{equation}
}

\laatikko{
Joskus murtolauseke sieventyy, jos sen esittääkin kahden murtoluvun summana.
\begin{equation}
\frac{ca+b}{c} = \frac{ca}{c} + \frac{b}{c} = a + \frac{b}{c}
\end{equation}
}

\laatikko{
Samantyyppiset asiat voidaan laskea yhteen tai \emph{ryhmitellä}.
\begin{equation}
ax^2 + bx + cx^2 + dy + ex = (a+b)x^2 + (b+c)x + dy
\end{equation}
}

\begin{esimerkki}

$ \frac{1}{6} + \frac{3}{2} = \frac{1}{2\cdot 3} + \frac{3}{2} = \frac{1}{2 \cdot 3} + \frac{3 \cdot 3}{2 \cdot 3} = \frac{1}{6} + \frac{9}{6} = \frac{10}{6} = \frac{\cancel{2} \cdot 5}{\cancel{2} \cdot 3} = \frac{5}{3}$

\end{esimerkki}

\begin{tehtava}
Sievennä
\begin{enumerate}
\item $\frac{2x^3}{x}$
\item $\frac{6x^2+8y}{2x^2}$
\item $ \frac{1-x}{3} + \frac{x+2}{6}$
\item $ \frac{5x-1}{3} - \frac{2x+5}{2}$
\end{enumerate}
\begin{vastaus}
\begin{enumerate}
\item $2x^2$
\item $3+4y$
\item $ -\frac{x}{6}$
\item $ \frac{2}{3} x - \frac{17}{6}$
\end{enumerate}
\end{vastaus}
\end{tehtava}


\input{02-lukualueet/11-juuret.tex}
\section{Neliöjuuri}

\laatikko{Luvun $a$ neliöjuuri on ei-negatiivinen luku, jonka neliö on $a$. Tämä voidaan ilmaista lyhyemmin $\sqrt{b^2}=b$.}

Neliöjuuren määritteleminen $\sqrt{a}^2=a$ ei johda samaan lopputulokseen. Pohdi, miksi näin on.
%%vai parempi antaa suoraan $\sqrt{a}^2=a$, kun $a \ge 0$
Jatkossa tälaisia määritelmän pieniä muokkauksia ja niistä aiheutuvia muutoksia olisi aina hyvä pohdiskella -- saattavat jopa auttaa muistamaan määritelmän oikean muodon.
%%%%%%%%%%%%%%% ONKO ITSEISARVO KÄSITELTY!!!!! %%%%%%%%%%%%%%%%%%%%%%%%%

%Määritelmäksi ei kelpaisi tämäkään. $\sqrt{a^2}=|a|$ EI OLE KÄSITELTY. Tulee esimerkkinä funktiosta funktioaiheen jälkeen.

Neliöjuurta ei siis nyt määritelty ollenkaan negatiivisille luvuille.

%yhtälöt tulevat vasta myöhemin, siksi esimerkit köyhiä

Esimerkki
\begin{align*}
\sqrt{4} = 2 \quad \textrm{, koska $2>0$ ja $2^2 =4$} 
\end{align*}

\todo{Tehtäviä, joissa pitää ottaa neliöjuuri jostain!}

\begin{tehtava}
% Huom! Vaatii Pythagoraan lauseen.
Taulutelevision kooksi (halkaisijaksi) on ilmoitettu mainoksessa $46,0$" \, ($116,8$ cm) ja kuvasuhteeksi 16:9. Kuinka leveä televisio on arviolta?
\begin{vastaus}
$40,7$" \, ($103,4$ cm).
\end{vastaus}
\end{tehtava}


\section{Kuutiojuuri}

\laatikko{Luvun $a$ kuutiojuuri on luku, jonka kuutio on $a$. Tämä voidaan ilmaista lyhyemmin $\sqrt[3]{b^3}=b$.
Määritelmäksi voisi ottaa myös $\sqrt[3]{b^3}=b$.}
%tämä jälkimmäinen on ehkä järkevämpi määritelmä.
%Olisi varmaan hyvä ottaa samanlainen määritelmä neliöjuuren tapauksessakin
%jolloin neliöjuureen määritelmään tulisi 2 ehtoa.
Kuutiojuuren voi siis ottaa mistä tahansa luvusta.
%vai reaaliluvusta?
%Reaaliluvuista puhutaan kuitenkin vasta myöhemmin, niin olkoon näin.

%\section{n. juuri}
Kaikkia juuria ei kuitenkaan kannata määritellä yksitellen. Tehdään siis mahdollisimman paljon kerralla. Edeltä kuitenkin voi huomata, että kuutiojuuri on määritelty kaikille luvuille, mutta neliöjuuri vain ei-negatiivisille luvuille. Tämä toistuu myös muissa juurissa. Määritellään siis parilliset ja parittomat juuret erikseen.

%on ehkä parempi esittää nämä molemma samalla eikä kuten subsections
%sama määritelmä, mutta todetaan parillisilla vaadittavan >= 0.
%%%RISTIRIITA ED. KANSSA
Juurimerkinnällä $\sqrt[n]{a}=b$ (luetaan \emph{ännäs juuri aasta on bee} tarkoitetaan lukua, joka toteuttaa ehdon $b^n = a$. Jotta juuri olisi ykskäsitteinen, on parillisilla juurilla ($\sqrt{a}, \sqrt[4]{a}, \sqrt[6]{a}$\ldots) vaadittava, että $b\ge0$.

%\subsection{parilliset juuret}

\laatikko{Luvun $a$ $n$.s juuri (luetaan \emph{ännäs juuri}) on ei-negatiivinen luku, jonka neliö on $a$. Tämä voidaan ilmaista lyhyemmin $\sqrt[n]{b^n}=b$.}

\subsection{Parittomat juuret}
\laatikko{Luvun $a$ n.s juuri on ei-negatiivinen luku, jonka neliö on $a$. Tämä voidaan ilmaista lyhyemmin $\sqrt[n]{b^n}=b$.}

Nyt on paikallaan todeta, että toista juurta $\sqrt[2]{a}$ merkitään $\sqrt{a}$.

\begin{tabular}{c|c}
parillinen juuri & pariton juuri\\
\hline
$\sqrt[n]{a}^n=a$, $a\ge0$ & $\sqrt[n]{a}^n$, kaikilla $a$
\end{tabular}

Esimerkiksi $\sqrt[3]{-8}=-2$ koska $(-2)^3=-8$, mutta $\sqrt[4]{-8}$ ei ole määritelty, koska minkään luvun neljäs potenssi ei ole negatiivinen.

%Mitä näille kahdelle seuraavalle tehdään?
%$\sqrt[n]{ab}=\sqrt[n]{a}\sqrt[n]{b}$
%Jos n on parillinen, niin on lisäksi vaadittava, että $a\ge0$ ja $b\ge0$.
%
%$\sqrt[n]{\frac{a}{b}}=\frac{\sqrt[n]{a}}{\sqrt[n]{b}}$
%Jos n on parillinen, niin on lisäksi vaadittava, että $a\ge0$ ja $b\ge0$.

\begin{tehtava}
Laske.
a) $\sqrt{64}$ \quad b) $\sqrt{-64}$ \quad c) $\sqrt[3]{64}$ \quad d) $\sqrt[3]{-64}$

\begin{vastaus}
a) 8 b) Ei määritelty c) 4 d) -4
\end{vastaus}
\end{tehtava}

\begin{tehtava}
Laske.
a) $\sqrt[4]{81}$ \quad b) $\sqrt[4]{-81}$ \quad c) $\sqrt[5]{32}$ \quad d) $\sqrt[5]{-32}$

\begin{vastaus}
a) 3 b) Ei määritelty c) 2 d) -2
\end{vastaus}
\end{tehtava}

\begin{tehtava}
Laske luvun $10$ potensseja: $10^1, 10^2, 10^3, 10^4, \ldots$ Kuinka monta nollaa on luvussa $10^n$? Laske sitten $\sqrt[6]{1~000~000}$ ja $\sqrt[10]{10~000~000~000}$.

\begin{vastaus}
$10^1 = 10, 10^2 = 100, 10^3 = 1~000, 10^4 = 10~000$. Luvussa $10^n$ on $n$ kappaletta nollia. Niinpä $\sqrt[6]{1~000~000} = 10$ ja $\sqrt[10]{10~000~000~000} = 10$.
\end{vastaus}
\end{tehtava}

\begin{tehtava}
Onko annettu juuri määritelty kaikilla luvuilla $a$? Millaisia arvoja juuri voi saada luvusta $a$ riippuen?\\
a) $\sqrt[4]{a^2}$ \quad b) $\sqrt[4]{-a^2}$ \quad c) $\sqrt[4]{(-a)^2}$ \quad d) $- \sqrt[4]{a^2}$

\begin{vastaus}

\begin{enumerate}[a)]
	\item Juuri on määritelty kaikilla luvuilla $a$, koska kaikkien lukujen neliöt ovat vähintään nolla. Vastaus on aina ei-negatiivinen.
	\item Juuri on määritelty vain luvulla $a = 0$. Muilla $a$:n arvoilla $-a^2$ on negatiivinen, jolloin parillinen juuri ei ole määritelty. Ainoa vastaus, joka voidaan saada, on siis $\sqrt[4]{0} = 0$.
	\item Juuri on määritelty kaikilla luvuilla $a$, koska $(-a)^2$ on aina vähintään nolla. Vastaus on aina ei-negatiivinen.
	\item Juuri on määritelty kaikilla luvuilla $a$, koska kaikkien lukujen neliöt ovat vähintään nolla. Vastaus on aina ei-positiivinen, koska $\sqrt[4]{a^2}$ on aina ei-negatiivinen.
\end{enumerate}
\end{vastaus}
\end{tehtava}

\begin{tehtava}
Onko annettu juuri määritelty kaikilla luvuilla $a$? Millaisia arvoja juuri voi saada luvusta $a$ riippuen?\\
a) $\sqrt[5]{a^2}$ \quad b) $\sqrt[5]{a^3}$ \quad c) $\sqrt[5]{-a^2}$ \quad d) $- \sqrt[5]{a^2}$

\begin{vastaus}

\begin{enumerate}[a)]
	\item Juuri on määritelty kaikilla luvuilla $a$. Vastaus on aina ei-negatiivinen.
	\item Juuri on määritelty kaikilla luvuilla $a$. Vastaus voi olla mikä tahansa luku.
	\item Juuri on määritelty kaikilla luvuilla $a$. Vastaus on aina ei-positiivinen.
	\item Juuri on määritelty kaikilla luvuilla $a$. Vastaus on aina ei-positiivinen.
\end{enumerate}
\end{vastaus}
\end{tehtava}
%\input{02-lukualueet/15-murtopotenssi.tex}
\input{02-lukualueet/16-irrationaaliluvut.tex}
\chapter{Reaaliluvut}

Jo antiikin aikoina huomattiin, että kaikki luvut eivät ole rationaalilukuja, eli kaikkia lukuja ei voi esittää kokonaislukujen suhteena. Tällaisia lukuja kutsutaan \emph{irrationaaliluvuiksi}.

Esimerkiksi $\sqrt{2}$ on irrationaaliluku. (Tämä todistetaan
luvun lopussa.) Toinen tuttu irrationaaliluku on $\pi$.

Rationaaliluvut eivät siis täytä lukusuoraa kokonaan, vaikka
rationaalilukuja onkin lukusuoralla tiheässä.

\missingfigure{Kuva lukusuorasta, johon on merkitty pii ja neliöjuuri 2}

Kun rationaalilukuihin otetaan mukaan irrationaaliluvut, saadaan reaalilukujen joukko $\mathbb{R}$. Kaikki rationaalilukuja koskevat
laskusäännöt pätevät myös reaaliluvuille.

Siinä missä rationaalilukujen desimaaliesitykset ovat päättyviä tai jaksollisia (tästä puhuttiin luvussa \ref{rationaaliluvut}), ovat
irrationaalilukujen desimaaliesitykset päättymättömiä ja
jaksottomia. Esimerkiksi luvun
\[\sqrt{2} = 1,414213562373095048801688724209\ldots\]
desimaaliesityksessä on toistaan muistuttavia kohtia
(tässäkin pätkässä 88 esiintyy kahdesti), mutta desimaalit eivät koskaan ''ala alusta''.

Reaalilukujen ominaisuuksien tarkka todistaminen on yllättävän
monimutkaista, ja se on tapana sivuuttaa lukiossa. Jatkossa
tyydymme toteamaan ilman todistusta, että rationaalilukujen
ominaisuudet yleistyvät myös reaaliluvuille.
Emme siis pysähdy pohtimaan, mitä esimerkiksi
$\pi \cdot \pi$ täsmälleen tarkoittaa.
Voimme silti laskea kyseiselle luvulle likiarvon
halutulla tarkkuudella,
\[ \pi\cdot \pi =9,86960440\ldots \].
Lisää reaalilukujen ominaisuuksista liitteessä \ref{aksioomat}.

Reaalilukujen myötä kaikki lukiossa käytettävät lukujoukot on nyt esitelty.
Ne on lueteltu seuraavassa:
\begin{center}\begin{tabular}{l|c|l}
Joukko & Symboli & Mitä ne ovat\\
\hline
Luonnolliset luvut & $\mathbb{N}$ &
Luvut 0, 1, 2, 3, $\ldots$ \\
Kokonaisluvut & $\mathbb{Z}$ & Luvut $\ldots$ -2, -1, 0, 1, 2 $\ldots$ \\
Rationaaliluvut & $\mathbb{Q}$ & Luvut, jotka voidaan esittää
murtolukuina \\
Reaaliluvut & $\mathbb{R}$ & Kaikki lukusuoran luvut
\end{tabular} \end{center}

\missingfigure{Kaavio, jossa N, Z, Q, R sisäkkäin.
Kussakin esimerkkiluku: 5, -2, 3/4, sqrt 2}

Lukualueita voidaan vielä tästäkin laajentaa. Seuraava laajennus olisi \emph{kompleksilukujen joukko} $\mathbb{C}$. Kompleksilukujen joukosta löytyy reaalilukujen lisäksi
esimerkiksi imaginaariyksikkö $i = \sqrt{-1}$.
%, jolle pätee $i^2=-1$. Minkään
%reaaliluvun neliö ei ole negatiivinen.
Kompleksiluvut eivät nykyään kuulu lukion oppimäärään, mutta
niitä tarvitaan esimerkiksi insinöörialoilla.
\input{02-lukualueet/18-kertaustiivistelma.tex}

\part{Funktiot ja yhtälöt}
\chapter{Yhtälö}
Monissa käytännön tilanteissa samalle asialle saadaan kaksi eri esitystapaa.

\begin{esimerkki}
Kuvassa oleva orsivaaka on tasapainossa. Toisessa vaakakupissa on kahden kilon siika ja toisessa puolen kilon ahven sekä tuntematon määrä lakritsia. Kuinka paljon vaakakupissa on lakritsia? (Ratkaistaan...) (Muita esimerkkejä, vähitellen vaikeutuvia (1. asteen) yhtälöitä)
\end{esimerkki}

\missingfigure{Kuva kaloista vaa'assa}

\laatikko{
\emph{Yhtälöksi} kutsutaan kahden lausekkeen merkittyä yhtäsuuruutta. Siis mielivaltaisille lausekkeille $A$ ja $B$ merkitään $A=B$. (Esim. $A=5x+\sqrt{x}$ ja $B=7x+7$). Jos yhtälön kummankin puolen lausekkeen arvot ovat samat, sanotaan että \emph{yhtälö pätee}.
}

Yhtälössä voi esiintyä \emph{muuttujia} eli symboleja, joiden arvoa ei ole etukäteen määrätty. Muuttujia merkitään usein kirjaimilla $x$, $y$ ja $z$. Niitä muuttujien arvoja, joilla yhtälö pätee, kutsutaan yhtälön \emph{ratkaisuiksi}. Yhtälön ratkaisemisella tarkoitetaan kaikkien yhtälön ratkaisujen selvittämistä.

\laatikko{
Eräs tapa ratkaista yhtälöitä on muokata niitä niin, että muokattu yhtälö pätee täsmälleen silloin kun alkuperäinen yhtälö pätee. Tällaisia sallittuja muunnoksia ovat esimerkiksi:
\begin{itemize}
\item Yhtälön molemmat puolet voidaan kertoa nollasta poikkeavalla luvulla $m$. Muutos tehdään aina molemmille puolille. Tällöin saadaan yhtälö $mA = mB$.
\item Yhtälön molemmille puolille voidaan lisätä tai molemmilta puolilta vähentää luku $n$. Tällöin saadaan yhtälö $A+n = B+n$.
\end{itemize}
}

\missingfigure{Kuva orvivaa'asta, jossa on myös heliumpallo}
% Nuo pitää ehkä perustella.

Monet yhtälöt ratkeavat toistamalla tällaisia muunnoksia kunnes yhtälö on niin yksinkertaisessa muodossa, että ratkaisu on helppo nähdä. Koska jokaisessa muokkausjonon yhtälössä ratkaisut ovat samat, näin saadaan alkuperäisen yhtälön ratkaisut.

%esimerkki tulee 1. asteen yhtälön yhteydessä

\laatikko{
Yhtälöt voidaan ratkaisujensa perusteella jakaa kolmeen tyyppiin:
\begin{enumerate}
\item Yhtälö, joka on aina tosi. Esimerkiksi yhtälöt $8=8$ ja $x=x$.
\item Yhtälö, joka on joskus tosi. Esimerkiksi yhtälö $x+4=7$ on tosi jos ja vain jos $x=3$.
\item Yhtälö, joka ei ole koskaan tosi. Esimerkiksi yhtälö $0=1$.
\end{enumerate}
}

Tämän kurssin ja ylipäätään matematiikan kannalta selvästi tärkein yhtälötyyppi on 2. Siirrymme nyt tarkastelemaan tärkeää erikoistapausta yhtälöistä, ensimmäisen asteen yhtälöitä.


%\chapter{Ensimmäisen asteen yhtälö}

\laatikko{
Ensimmäisen asteen yhtälö on yhtälö, joka on esitettävissä muodossa $ax+b=0$, jossa $a \neq 0$.
}

\begin{theorem}
Kaikki muotoa $ax+b=cx+d$ olevat yhtälöt, joissa $a \neq c$, ovat ensimmäisen asteen yhtälöitä.
\end{theorem}

\begin{proof}
\begin{align*}
ax+b &= cx+d & &| \, \textbf{Vähennetään molemmilta puolilta $cx+d$.} \\
ax+b - (cx+d) &= 0 & &| \, \textbf{Järjestellään termejä uudelleen.} \\
ax - cx + b - d &= 0 & &| \, \textbf{Otetaan yhteinen tekijä.} \\
(a-c)x + (b-d) &= 0 & &| \, \textbf{Tämä on määritelmän mukainen ensimmäisen asteen yhtälö, koska $a \neq c$.}
\end{align*}
\end{proof}

\begin{theorem}
Yleinen lähemistymistapa muotoa $ax+b = cx+d$ olevien yhtälöiden ratkaisuun on: \\
(1) Vähennä molemmilta puolilta $cx$. Saat yhtälön $(a-c)x + b = d$. \\
(2) Vähennä molemmilta puolita $b$. Saat yhtälön $(a-c)x = d-b$. \\
(3) Jaa $(a-c)$:llä. Saat yhtälön ratkaistuun muotoon $x = \frac{d-b}{a-c}$.
\end{theorem}

Esimerkki. Yhtälön $7x+4=4x+7$ ratkaisu saadaan seuraavasti:
\begin{align*}
7x+4 &= 4x+7 & &| \, \text{Vähennetään molemmilta puolilta 4x.} \\
3x+4 &= 7 & &| \, \text{Vähennetään molemmilta puolilta 4.} \\
3x &= 3 & &| \, \text{Jaetaan molemmat puolet kolmella eli kerrotaan $\frac{1}{3}$:lla.} \\
x &= 1 & &| \, \text{Saimme yhtälön ratkaistuun muotoon. $x=1$ on siis yhtälön ratkaisu.} \\
\end{align*}

\begin{tehtava}
%
Ratkaise:
\begin{enumerate}[a)]
\item $x + 4 = 5$
\item $1 - x = -3$
\item $7x = 35$
\item $-2x = 4$
\item $10 - 2x = x$
\item $9x + 4 = 6 - x$
\item $\frac{2x}{5} = 4$
\item $\frac{x}{3} + 1 = \frac{5}{6} - x$
\end{enumerate}
\begin{vastaus}
\begin{enumerate}[a)]
\item $x=1$
\item $x=4$
\item $x=35/7$
\item $x=-2$
\item $x=10/3$
\item $x=1/5$
\item $x=10$
\item $x=-1/8$
\end{enumerate}
\end{vastaus}
\end{tehtava}

\begin{tehtava}
Ratkaise kysytty muuttuja yhtälöstä
\begin{enumerate}[a)]
\item $F=ma$, $m=?$ "voima on massa kertaa kiihtyvyys"
\item $p=\frac{F}{A}$, $F=?$ "paine on voima jaettuna alalla"
\item $A=\pi r^2$, $r=?$ "(pallon) pinta-ala on pii kertaa säde toiseen"
\item $V=\frac{1}{3} \pi r^2 h$, $h=?$ "(kartion) tilavuus on kolmasosa pii kertaa säde toiseen kertaa korkeus"
\end{enumerate}
\begin{vastaus}
\begin{enumerate}[a)]
\item $m=\frac{F}{a}$
\item $F=p A$
\item $r=\sqrt{\frac{A}{\pi}}$
\item $h=\frac{V}{ \frac{1}{3} \pi r^2 h}$
\end{enumerate}
\end{vastaus}
\end{tehtava}

\begin{tehtava}
Kännykkäliittymän kuukausittainen perusmaksu on 2,90 euroa. Lisäksi jokainen puheminuutti ja tekstiviesti maksaa 0,69 senttiä. Pekan kännykkälasku kuukaudelta oli 27,05 euroa.

\begin{enumerate}[a)]
	\item Kuinka monta puheminuuttia/tekstiviestiä Pekka käytti kuukauden aikana?
	\item Pekka lähetti kaksi tekstiviestiä jokaista viittä puheminuuttia kohden. Kuinka monta tekstiviestiä Pekka lähetti?
\end{enumerate}

	\begin{vastaus}
		\begin{enumerate}[a)]
			\item 350 puheminuuttia/tekstiviestiä
			\item 100 tekstiviestiä
		\end{enumerate}
	\end{vastaus}
\end{tehtava}

\begin{tehtava}
Sadevesikeräin näyttää vesipatsaan korkeuden millimetreissä. Eräänä aamuna kello 9 keräimessä oli 5 mm vettä. Seuraavana aamuna samaan aikaan keräimessä oli 23 mm vettä. Muodosta yhtälö ja ratkaise, kuinka paljon vettä oli satanut keskimäärin tunnissa kuluneen vuorokauden aikana.
	\begin{vastaus}
	0,75 mm/tunti
	\end{vastaus}
\end{tehtava}
\chapter{Erilaisia funktioita}

Tässä kappaleessa esitellään yleisiä funktioita, niiden ominaisuuksia ja käyttösovelluksia.

\section{}
\chapter{Funktio}
\laatikko{Funktio on yksi tärkeimmistä matemaattisista työkaluista ja sen ymmärtäminen on tärkeää.}

Matematiikassa tutkitaan paljon asioiden välisiä riippuvuuksia. Tällaiset riippuvuudet voidaan muotoilla funktioiden avulla. Esimerkiksi tuotteen arvonlisäveroprosentti riippuu tuoteryhmästä. Tämä riippuvuus voidaan kirjoittaa funktiona eri tuotteiden joukolta $A$ reaalilukujen joukolle $\mathbb{R}$, missä funktio liittää jokaiseen tuotteeseen sen arvonlisäveroprosentin.

[Esimerkki, kuva arvonlisäverofunktiosta, missä \[A = \{\text{ahvenfilee}, \text{AIV-rehu}, \text{auto}, \text{runokirja}, \text{ravintola-ateria}, \text{särkylääke}, \text{televisio}\},\]$f(\text{ahvenfilee}) = 13$, $f(\text{AIV-rehu}) = 13$, $f(\text{auto}) = 23$, $f(\text{runokirja}) = 9$, $f(\text{ravintola-ateria}) = 13$, $f(\text{särkylääke}) = 9$, $f(\text{televisio}) = 23$]

\laatikko{Funktio $f$ joukosta $A$ joukkoon $B$ on sääntö, joka liittää $A$:n jokaiseen alkioon $x$ täsmälleen yhden $B$:n alkion, jota merkitään $f(x)$. $A$:ta kutsutaan funktion $f$ määrittelyjoukoksi ja $B$:tä funktion $f$ maalijoukoksi.}

Käytäntöjä:
\begin{itemize}
\item $f(x) = y$ lausutaan: "Funktio saa arvon y pisteessä x."
\item Funktioita voidaan kutsua myös kuvauksiksi.
\item Funktion määrittely- ja maalijoukot jätetään usein merkitsemättä, jos ne ovat selviä asiayhteydestä. Tällä kurssilla maalijoukko on yleensä reaaliluvut.
\item Funktion sääntöä kutsutaan usein pelkästään funktioksi.
\end{itemize}

Funktion arvojoukko koostuu niistä maalijoukon alkioista, jotka funktio saa arvokseen ainakin yhdessä määrittelyjoukon pisteessä.

Funktion sääntö määritellään usein kaavan avulla. Voimme esimerkiksi määritellä funktion $f$ reaaliluvuilta reaaliluvuille kaavalla
\[f(x) = x^2 + 1,\]
eli $f$ liittää jokaiseen reaalilukuun $x$ luvun $x^2+1$. Tästä funktiosta huomaamme, että arvojoukko ei ole aina sama kuin maalijoukko. Esimerkiksi luku $0$ kuuluu funktion $f$ maalijoukkoon, mutta ei funktion $f$ arvojoukkoon, koska $x^2+1\geq 1$ neliön epänegatiivisuuden perusteella. [kuva]

\esimerkki{Määritellään funktio $f$ kaavalla [kuva]
\[f(x) = \frac{1}{x-1}.\]
Funktiolle ei ole erikseen annettu määrittelyjoukkoa, joten se täytyy päätellä asiayhteydestä. Huomataan, että sääntö ei määrittele funktion $f$ arvoa kun $x = 1$, mutta kylläkin kaikilla $x\in\mathbb{R}\setminus\{1\}$ [vai?]. Näin ollen luonnollinen valinta määrittelyjoukoksi on $\mathbb{R}\setminus\{1\}$. Voimme myös selvittää funktion arvojoukon kiinnitettyämme määrittelyjoukon. Valitulla määrittelyjoukolla $\mathbb{R}\setminus\{1\}$ voimme muodostaa yhtälön $f(x)=a$ jollekin reaaliluvulle a. Tutkitaan, milloin tällä yhtälöllä on ratkaisu(ja).
\begin{align*}
a &= \frac{1}{x-1} & &| \, \text{Oletamme, että $x \neq 1$, jolloin voimme kertoa $(x-1)$:llä puolittain.} \\
a(x-1) &= 1 \\
x-1 &= \frac{1}{a} \\
x &= 1+\frac{1}{a} & &| \, \text{Huomaamme, että saamme ratkaisun kaikilla $a \neq 0$.} \\
& & &| \, \text{Yhtälö ei ole hyvinmääritelty, kun $a = 0$.}
\end{align*}
}
\begin{esimerkki}

jokin riippuu jostakin, esitä määrittely ja arvojoukko muodossa f:A->B
\end{esimerkki}

Funktion nimeäminen, f(x) vastaa merkinnältään sin(x):ää. jälkimmäisen tunnistaa, tuttu funktio

\section{Usean muuttujan funktiot}

\laatikko{Funktiolla voi olla monta muuttujaa. Jos funktion $f$ arvo riippuu esimerkiksi muuttujista $x$ ja $y$, merkitään funktiota $f(x,y)$}

\begin{esimerkki}

Olkoon suorakulmion korkeus $x$ ja leveys $y$. Tiedetään, että suorakulmion pinta-ala voidaan laskea kertomalla sen korkeus ja leveys keskenään, joten suorakulmion pinta-ala voidaan esittää tunnettujen pituuksien $x$ ja $y$ funktiona: $A(x,y)=xy$.

Pituudet voivat olla vain positiivisia reaalilukuja: $x,y$

 määrittely ja arvojoukko
\end{esimerkki}



\input{03-funktiot-ja-yhtalot/05-kertaustiivistelma.tex}


\part{Sovelluksia}
\chapter{Verrannollisuus}

Verrannollisuudella tarkoitetaan tilannetta, jossa 

Suoraan verrannollisuus tarkoittaa, että kahden asian suhde pysyy vakiona. Jos toinen kaksinkertaistuu, kaksinkertaistuu toinenkin. Esimerkiksi kaupasta ostettujen hedelmien määrä ja kauppahinta ovat suoraan verrannollisia toisiinsa. Jos ostat kaksi kertaa enemmän banaaneja, joudut myös maksamaan kaksi kertaa enemmän. Hinnan ja ostettujen banaanien massan\footnote{Arkikielessä puhutaan yleensä painosta.} suhde on molemmissa tapauksissa vakio. Hedelmäesimerkin tapauksessa tätävakiota kutsutaan kilohinnaksi.

Matemaattisesti suoraan verrannollisuus merkitään seuraavasti. Jos suure $a$ on suoraan verrannollinen suureeseen $b$, merkitään

\begin{equation}
    \frac{a}{b}=c,
\end{equation}
missä $c$ on vakio.

Kääntäen verrannollisuus tarkoittaa, että kahden asian tulo pysyy vakiona. Jos toinen kaksinkertaistuu, toinen puolittuu. Esimerkiksi nopeus ja matkaan tarvittava aika ovat kääntäen verrannollisia toisiinsa. Jos ajat koulumatkan kaksi kertaa nopeammin, matka-aika puolittuu.

Muita esimerkkejä kääntäen verrannollisuudesta ovat:
\begin{itemize}
    \item .
\end{itemize}

Matemaattisesti kääntäen verrannollisuus merkitään seuraavasti. Jos suure $a$ on kääntäen verrannollinen suureeseen $b$, merkitään
\begin{equation}
    ab=c,
\end{equation}
missä $c$ on vakio.



\chapter{Verrannollisuus: sovelluksia}

\begin{tehtava}
    % Lyhyt matikka 1, s. 72
    Pohdi, kuinka toinen suure muuttuu, kun toinen suure kaksinkertaistuu,
    kolminkertaistuu, puolittuu jne. Ovatko suureet suoraan verrannolliset?
    
    \begin{enumerate}
        \item kuljettu matka ja kulunut aika, kun keskinopeus on 30 km/h
        \item kananmunien lukumäärä ja niiden kovaksi keittämiseen tarvittava keittoaika
        \item hedelmätiskiltä valitun vesimelonin paino ja hinta
        \item neliön sivun pituus ja neliön pinta-ala
    \end{enumerate}
    
    \begin{vastaus}
        Vastaus:
        \begin{enumerate}
            \item Ovat.
            \item Eivät ole.
            \item Ovat.
            \item Eivät ole, sillä esimerkiksi kun neliön sivun pituus
                kaksinkertaistuu 1 cm:stä 2 cm:iin, niin neliön pinta-ala
                nelinkertaistuu 1 cm$^2$:stä 4 cm$^2$:iin.
        \end{enumerate}
    \end{vastaus}
\end{tehtava}

\begin{tehtava}
Ratkaise
\begin{enumerate}
\item $ \frac{7}{x} = \frac{16}{x}$
\item $ \frac{x}{3} = \frac{1}{7}$
\end{enumerate}
\begin{vastaus}
\begin{enumerate}
\item $x= \frac{7}{2}$
\item $x= \frac{3}{7}$
\end{enumerate}
\end{vastaus}
\end{tehtava}

\begin{tehtava}
Rento pyöräilyvauhti kaupunkiolosuhteissa on noin $20$ km/h. Lukiolta urheiluhallille on matkaa $7$ km. Kuinka monta minuuttia kestää arviolta pyöräillä lukiolta urheiluhallille?
\begin{vastaus}
Viiden minuutin tarkkuudella $20$ min.
\end{vastaus}
\end{tehtava}

\begin{tehtava}
    Isi ja lapset ovat ajamassa mökille Sotkamoon. Ollaan ajettu jo neljä
    viidennestä matkasta ja aikaa on kulunut kaksi tuntia. ''Joko ollaan perillä?''
    kysyvät lapset takapenkiltä. Kuinka pitkään vielä arviolta kuluu, ennen
    kuin ollaan mökillä?
    
    \begin{vastaus}
        Vastaus: 1 h 15 min
    \end{vastaus}
\end{tehtava}

\begin{tehtava}
    Äidinkielen kurssilla annettiin tehtäväksi lukea eräs 300-sivuinen romaani.
    Eräs opiskelija otti aikaa ja selvitti lukevansa vartissa seitsemän sivua.
    Kuinka monta tuntia häneltä kuluu arviolta koko romaanin lukemiseen, jos
    taukoja ei lasketa?
    
    \begin{vastaus}
        Vastaus: 642 minuuttia eli 10 h 42 min.
    \end{vastaus}
\end{tehtava}

\chapter{Prosenttilaskenta}

Sana prosentti tulee latinan kielen sanoista pro centum, mikä tarkoittaa
kirjaimellisesti sataa kohden. Prosentteja käytetään ilmaisemaan suhteellista
osuutta. Lukua, josta suhde lasketaan, kutsutaan \emph{perusarvoksi}. Prosentin
merkki on \%. Esimerkiksi jos sadan euron hintaisesta tuotteesta on alennettu 25
prosenttia, niin tuotteen alennettu hinta on 75 euroa. Jos sen sijaan alkuperäinen
hinta nousee 15 prosenttia, niin tuotteen uusi hinta on 115 euroa. Perusarvo on
molemmissa tapauksissa 100 euroa.

\laatikko{1 prosentti $= 1 \% = \frac{1}{100} = 0,01$}

\laatikko{Esimerkki: \\$6 \% = \frac{6}{100} = 0,06$, $48,2 \% = \frac{48}{100} = 0,482$, $140 \% = \frac{140}{100} = 1,40$}

Suhdeluku muutetaan prosenteiksi kertomalla se luvulla 100 ja lisäämällä
lopputuloksen jälkeen prosenttimerkki.

\begin{esimerkki}
Vesa ansaitsee kuukaudessa 2300 euroa ja Antero 1700 euroa.
    Kuinka monta prosenttia Anteron tulot ovat Vesan tuloista? 
    
    {\bf Ratkaisu.}
    
    Lasketaan
    \[
    \frac{1700}{2300} \cdot 100 \% \approx 0,739\cdot 100 \% = 73,9 \%.
    \]
    Laskuissa käytettävä perusarvo on Vesan palkka eli 2300 euruoa.
    
    {\bf Vastaus.}
     $73,9 \%$
\end{esimerkki}


Vertailuprosentilla ilmaistaan, kuinka paljon toinen luku on suurempi kuin toinen. Vertailukohteena käytetään aina sitä lukua, johon verrataan. Jos siis halutaan tietää, kuinka monta prosenttia luku $a$ on suurempi kuin $b$, vertailuprosentti saadaan laskettua kaavalla
\[
\frac{a-b}{b} \cdot 100 \%.
\]

\begin{esimerkki}
    Vesa ansaitsee kuukaudessa 2300 euroa ja Antero 1700 euroa.
    Kuinka monta prosenttia enemmän Vesa ansaitsee kuin Antero?
    
    {\bf Ratkaisu.}
    
    Lasketaan aluksi Vesan ja Anteron palkkojen erotus
    \[
    2300-1700 = 600.
    \]
    Sitten lasketaan kuinka monta prosenttia 600 euroa on Anteron palkasta:
    \[
    \frac{600}{1700} \cdot 100 \% \approx 0,353\cdot 100\% = 35,3 \%.
    \]
    
    {\bf Vastaus.}
    $35,3 \%$
\end{esimerkki}

Prosentteja käytetään usein ilmaisemaan suureiden muutoksia. Muutosprosenttia laskettaessa perusarvona on alkuperäinen arvo, johon nähden muutos on tapahtunut.

\begin{esimerkki}
    Vesan paino on tammikuussa 68 kg ja kesäkuussa 64 kg. Kuinka monta prosenttia Vesa on laihtunut?

    {\bf Ratkaisu.}

    Lasketaan 
    \[
    \frac{68-64}{68}\cdot 100\% = \frac{4}{68} \cdot 100\%=0,059\cdot 100\% =5,9\%.
    \]
    
    {\bf Vastaus.}
    Vesa on laihtunut $5,9\%$.
\end{esimerkki}


Prosenttiyksikkö mittaa prosenttiosuuksien välisiä eroja. Jos prosenttiluku muuttuu, muutos voidaan ilmaista joko prosentteina tai prosenttiyksikköinä.


\begin{esimerkki}
    Tuotteen markkinaosuus on vuoden tammikuussa 10 \% ja kesäkuussa 15 \%. 
    \begin{enumerate}[a)]
    \item Kuinka monta prosenttia tuotteen markkinaosuus on noussut?
    
    \item Kuinka monta prosenttiyksikköä tuotteen markkinaosuus on noussut?
    \end{enumerate}
    
    {\bf Ratkaisu.} 
    
    \begin{enumerate}[a)]
    \item Tuotteen markkinaosuus on noussut
    \[
    \frac{15-10}{10} \cdot 100 \%= \frac{5}{10}\cdot 100\% = 50\%.
    \]
    
    \item Tuotteen markkinaosuus on noussut $15-10=5$ prosenttiyksikköä. 
    \end{enumerate}
    
    {\bf Vastaus.}
    
    \begin{enumerate}[a)]
    \item 50 prosenttia
    \item 5 prosenttiyksíkköä.
    \end{enumerate}
\end{esimerkki}




\section{Perusprosenttilaskut}

\begin{itemize}
	\item Prosenttiluvun laskeminen
	\item Prosenttiarvon laskeminen
	\item Perusarvon laskeminen
\end{itemize}

\section{Vertailu prosenttien avulla}

\begin{itemize}
	\item Muutosprosentti, vertailuprosentti
	\item Prosentuaalinen muutos
	\item Prosenttiyksikkö
\end{itemize}

\section{Prosenttiyhtälöitä ja sovelluksia}

\begin{tehtava}
    Laukku maksaa 225 \euro ja on 25\%:n alennuksessa. Paljonko alennettu hinta on?
    
    \begin{vastaus}
    Vastaus: 168,75 \euro
    \end{vastaus}
\end{tehtava}

\begin{tehtava}
    %Pyramidi 1, s. 80
    Kirjan myyntihinta, joka sisältää arvolisäveron, on 8\% suurempi kuin kirjan
    veroton hinta. Laske kirjan veroton hinta, kun myyntihinta on 15\euro.
    
    \begin{vastaus}
        Vastaus: 13,89 \euro
    \end{vastaus}
\end{tehtava}

\begin{tehtava}
    Perussuomalaisten kannatus oli vuoden 2007 eduskuntavaaleissa 4,1\% ja
    vuoden 2011 eduskuntavaaleissa 19,1\%. Kuinka monta prosenttiyksikköä kannatus nousi? Kuinka monta prosenttia kannatus nousi?
    \begin{vastaus}
    Vastaus: Kannatus nousi 15 \%-yksikköä ja 365,9 \%.
    \end{vastaus}
\end{tehtava}

\begin{tehtava}
    Askartelukaupassa on alennusviikot, ja kaikki tavarat myydään 60\%n alennuksella. Viimeisenä päivänä kaikista hinnoista annetaan 
    vielä lisäalennus, joka lasketaan aiemmin alennetusta hinnasta. Minkä suuruinen lisäalennus tulee antaa, jos lopullisen 
    kokonaisalennuksen halutaan olevan 80\%?

    \begin{vastaus}
        Vastaus: 50\%.
    \end{vastaus}
\end{tehtava}

\begin{tehtava}
    %tässä tehtävässä pitää tietää potenssi
    Erään pankin myöntämä opintolaina nousee korkoa 2\% vuodessa. Kuinka monta prosenttia laina on noussut korkoa alkuperäiseen 
    summaan verrattuna kymmenen vuoden kuluttua?

    \begin{vastaus}
        Vastaus: 22\%.
    \end{vastaus}
\end{tehtava}

\begin{tehtava}
Yleinen arvonlisäveroprosentti oli Suomessa vuonna 2012 23 \% tuotteen verottomasta
hinnasta. Kaupassa maksetteva hinta koostuu verottomasta hinnasta
ja verosta. Kuinka monta prosenttia vero on tuotteen
myyntihinnasta?
\begin{vastaus}
18,0 \%
\end{vastaus}
\end{tehtava}

Ansiotuloverotus on Suomessa progressiivista: suuremmista tuloista maksetaan

\begin{tehtava}
    Tuoreissa omenissa on vettä 80\% ja sokeria 4\%. Kuinka monta prosenttia sokeria on samoissa omenissa, kun ne on kuivattu siten, 
    että kosteusprosentti on 20? [K2000, 4]
    
    \begin{vastaus}
        Vastaus: 16\%
    \end{vastaus}
\end{tehtava}

\begin{tehtava}
    Kappaleen vapaa pudotus korkeudelta $x$ maahan on kääntäen verrannollinen putoamiskiihtyvyyden $g$ neliöjuureen. $g$ on kullekin     
    taivaankappaleelle ominainen ja eri puolilla samaa taivaankappaletta likimain vakio. Empire State Buildingin katolta (korkeus 
    $381$ m) pudotetulla kuulalla kestää n. $6,2$ s osua maahan. Marsin putoamiskiihtyvyys on $37,6$\%  Maan putoamiskiihtyvyydestä. 
    Jos Empire State Building sijaitsisi Marsissa, kuinka monta prosenttia pitempi aika kuluisi kuulan maahan osumiseen?

    \begin{vastaus}
        Vastaus: $10,1$ s
    \end{vastaus}
\end{tehtava}

\input{04-sovelluksia/05-kertaustiivistelma.tex}

\part{Kertaus ja harjoituskokeita}
\chapter{Verrannollisuus}

Verrannollisuudella tarkoitetaan tilannetta, jossa 

Suoraan verrannollisuus tarkoittaa, että kahden asian suhde pysyy vakiona. Jos toinen kaksinkertaistuu, kaksinkertaistuu toinenkin. Esimerkiksi kaupasta ostettujen hedelmien määrä ja kauppahinta ovat suoraan verrannollisia toisiinsa. Jos ostat kaksi kertaa enemmän banaaneja, joudut myös maksamaan kaksi kertaa enemmän. Hinnan ja ostettujen banaanien massan\footnote{Arkikielessä puhutaan yleensä painosta.} suhde on molemmissa tapauksissa vakio. Hedelmäesimerkin tapauksessa tätävakiota kutsutaan kilohinnaksi.

Matemaattisesti suoraan verrannollisuus merkitään seuraavasti. Jos suure $a$ on suoraan verrannollinen suureeseen $b$, merkitään

\begin{equation}
    \frac{a}{b}=c,
\end{equation}
missä $c$ on vakio.

Kääntäen verrannollisuus tarkoittaa, että kahden asian tulo pysyy vakiona. Jos toinen kaksinkertaistuu, toinen puolittuu. Esimerkiksi nopeus ja matkaan tarvittava aika ovat kääntäen verrannollisia toisiinsa. Jos ajat koulumatkan kaksi kertaa nopeammin, matka-aika puolittuu.

Muita esimerkkejä kääntäen verrannollisuudesta ovat:
\begin{itemize}
    \item .
\end{itemize}

Matemaattisesti kääntäen verrannollisuus merkitään seuraavasti. Jos suure $a$ on kääntäen verrannollinen suureeseen $b$, merkitään
\begin{equation}
    ab=c,
\end{equation}
missä $c$ on vakio.


\input{05-loppuosa/02-yo-tehtavia.tex}

% Tähän tulee liitteitä
% Esimerkiksi loogiset symbolit, reaalilukujen aksioomat, kompleksilukuintro, ...
\part{Liiteet}
% Vaihda tähän kirjaimin kulkeva "numerointi"
\chapter{Logiikka ja joukko-oppi}
\chapter{Reaalilukujen aksioomat}
Reaaliluvut ovat kunta, eräs algebrallinen rakenne. Myös esimerkiksi rationaaliluvut ja seuraavassa liitteessä esiteltävät kompleksiluvut muodostavat kunnan. Sen sijaan luonnolliset luvut ja kokonaisluvut eivät ole kuntia.

Reaalilukujen aksiomaattinen määritelmä muodostuu kolmesta osasta:

\textbf{1. Kunta-aksioomat reaalilukuihin sovellettuna} \\
\begin{align*}
&\text{K1.} \, \forall x, y \in \mathbb{R}: x+(y+z) = (x+y)+z & &| \, \text{summan liitäntälaki} \\
&\text{K2.} \, \exists 0 \in \mathbb{R}: x+0 = x & &| \, \text{summan neutraalialkio} \\
&\text{K3.} \, \forall x \in \mathbb{R} \, \exists (-x) \in \mathbb{R}: x+(-x)=0 & &| \, \text{vasta-alkio} \\
&\text{K4.} \, \forall x, y \in \mathbb{R}: x+y = y+x & &| \, \text{summan vaihdantalaki} \\
&\text{K5.} \, \forall x, y, z \in \mathbb{R}: x*(y+z) = x*y + x*z & &| \, \text{osittelulaki} \\
&\text{K6.} \, \forall x, y, z \in \mathbb{R}: x*(y*z) = (x*y)*z & &| \, \text{tulon liitäntälaki} \\
&\text{K7.} \, \exists 1 \in \mathbb{R}: 1*x = x & &| \, \text{tulon neutraalialkio} \\
&\text{K8.} \, \forall x \in \mathbb{R} \setminus \{0\} \, \exists x^{-1} \in \mathbb{R} \setminus \{0\}: x*x^{-1}=1 & &| \, \text{tulon käänteisalkio} \\
&\text{K9.} \, \forall x, y \in \mathbb{R}: x*y = y*x & &| \, \text{tulon vaihdantalaki}
\end{align*}


\end{document}