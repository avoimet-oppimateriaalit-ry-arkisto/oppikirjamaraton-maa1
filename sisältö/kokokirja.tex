

% Kun käytät tätä, älä lataa pakettia stfloats tai fix2col. Tämä lataa
% myös paketin fixltx2e.
%% "When the new output routine for LaTeX3 is done, this package will
%% be obsolete. The sooner the better..."
\RequirePackage{dblfloatfix}
% Kun käytät tätä, et enää tarvitse paketteja type1cm ja type1ec:
\RequirePackage{fix-cm}

% Helpottaa pdf(la)tex:in läsnäolon tarkistusta if-lauseilla
\RequirePackage{ifpdf}

\documentclass[a4paper,onecolumn,12pt,finnish,oneside,final]{boek3}

\setcounter{secnumdepth}{7}

%%%%%%%%%%%%%%%%%%%%%%%%%%%%%%%%%%%%%%%%%%%%%%%%%%%%%%%%%%%%%%%%%%%%%%%%%
% Suometus ja fontit
\usepackage[utf8]{inputenc}

% cmap: "Make the PDF files generated by pdflatex "searchable and copyable"
% in Adobe (Acrobat) Reader and other compliant PDF viewers."
% (Pakko olla ennen fontenc-pakettia)
\ifpdf % Eräillä saattaa olla buginen cmap. Yritetään välttää sitä.
\usepackage{cmap}
\fi

\usepackage[T1]{fontenc}
% Suometusten asettaminen jatkuu myöhemmin

\usepackage{blindtext}
\usepackage{lipsum}

%%%%%%%%%%%%%%%%%%%%%%%%%%%%%%%%%%%%%%%%%%%%%%%%%%%%%%%%%%%%%%%%
% Fontit
%

% Yleisimmät erikoismerkit, kuten copyleft
\usepackage{textcomp}

%%% Matikkafontteja

% Matikka-symboleita
\usepackage{latexsym}

% AMS:sän matikka-paketteja:
% Lisätietoa paketista optiolla "?"
%\usepackage[?]{amsmath}
\RequirePackage{amsmath}
\RequirePackage{amsfonts}
\RequirePackage[psamsfonts]{amssymb}
\RequirePackage{amsxtra}
\RequirePackage{amscd}
\RequirePackage{amsthm}
\RequirePackage{mathrsfs}

% Math fonts 
\usepackage{arevmath}

% Text fonts 
% TODO: Poistettu jotta kääntyisi kaikilla.
%\usepackage{dejavu}

\renewcommand{\familydefault}{\sfdefault}
\fontfamily{\familydefault}

\usepackage{titlesec}

%\titleformat{\part}[display]
%  {\normalfont\Huge\filright\bfseries}{\MakeUppercase{Osa\ \thepart}}{\Huge\filright\MakeUppercase}

\titleformat{\part}
  {\normalfont\filright\Huge\bfseries\rmfamily}{Osa \thepart}{1em}{}

%\titleformat{\chapter}[display]
%  {\normalfont\huge\filright\bfseries}{\MakeUppercase{\chaptertitlename\ \thechapter}}{\Huge\filright\MakeUppercase}

\titleformat{\chapter}
  {\normalfont\filright\huge\bfseries\rmfamily}{\chaptertitlename\ \thechapter}{1em}{}

\titleformat{\section}
  {\normalfont\filright\LARGE\bfseries\rmfamily}{\thesection}{1em}{}

\titleformat{\subsection}
  {\normalfont\filright\Large\bfseries\rmfamily}{\thesubsection}{1em}{}

\titleformat{\subsubsection}
  {\normalfont\filright\large\bfseries\rmfamily}{\thesubsubsection}{1em}{}

\titleformat{\paragraph}[runin]
  {\normalfont\normalsize\bfseries\rmfamily}{\theparagraph}{1em}{}

\titleformat{\subparagraph}[runin]
  {\normalfont\normalsize\bfseries\rmfamily}{\thesubparagraph}{1em}{}




% Tukea pdf(la)texin microtypes-laajennoksille. Pakko olla
% fonttimäärityksien jälkeen.
\ifpdf
\usepackage[protrusion=true,expansion=true,verbose=true]{microtype}
\else
\usepackage[verbose=true]{microtype}
\fi


\makeatletter
\g@addto@macro\verbatim{\pdfprotrudechars=0 \pdfadjustspacing=0\relax}
\makeatother




% Fontit asetettu!

% Loput suometukset (mutta myös enkunkieltä saatetaan tarvita):
\usepackage[finnish,english]{babel}
%\usepackage[finnish]{babel}

% Suometukset asetettu!

% if-then-else-rakenteita helposti
\usepackage{ifthen}

% Lisää kokomäärityksiä:
\usepackage[12pt]{moresize}

% Tee jotain jokaisen sivun kohdalla:
% (totpages tarvii tätä)
\usepackage{everyshi}

% Värien tuki:
% (Mihinkähän mä tätä tarvitsinkaan?)
%\usepackage{color}

% Kuvien lisäys dokumenttiin:
% (Mihinkähän mä tätä tarvitsinkaan?)
%\usepackage{graphicx}

% Laskutehtävien suoritusta (geometry-paketti hyötyy tästä):
\usepackage{calc}


\usepackage[nomarginpar,includeheadfoot]{geometry}

% vasen marginaali
\geometry{lmargin=4.0cm}
% oikea marginaali
\geometry{rmargin=2.0cm}
% ylämarginaali
\geometry{tmargin=2.0cm}
% alamarginaali
\geometry{bmargin=2.0cm}


%%%%%%%%%%%%%%%%%%%%%%%%%%%%%%%%%%%%%%%%%%%%%%%%%%%%%%%%%%%%%%%%
% Ykkösvälike, puolitoistakertainen välike ja kakkosvälike yms.
\usepackage{setspace}
%\singlespacing
\onehalfspacing
%\spacing{1.7} % Tämä bugasi joskus aiheuttaen tällaista: (\end occurred inside a group at level 1) 
%\doublespacing

% URL:ien ladonta ja "tavutus"
\usepackage[obeyspaces,spaces,hyphens,T1]{url}
%\renewcommand\url{\begingroup \def\UrlLeft{<URL: }\def\UrlRight{>}\urlstyle{rm}\Url}
%\def\UrlBigBreakPenalty{120}
%\def\UrlBreakPenalty{200}

% Helpottaa TeX:iin liittyvien nimien ladontaa (LaTeX, BibTeX jne.)
\usepackage{texnames}

%%%%%%%%%%%%%%%%%%%%%%%%%%%%%
% Upeet headerit ja footerit





% Ei sisennetä kappaleen ekaa riviä
\setlength{\parindent}{0.0cm}


\ifpdf % We are running pdftex
\pdfcompresslevel=9
\pdfpkresolution=1200
\fi

%\theoremstyle{definition}
\newtheorem{theorem}{Teoreema}

% Infoboksi keltaisella taustalla
\usepackage{mdframed}
\newcommand{\laatikko}[1]{\begin{mdframed}[backgroundcolor=yellow] #1 \end{mdframed}}

% Esimerkeille oma komento http://tex.stackexchange.com/questions/21227/example-environment
\newtheorem*{esimerkki}{Esimerkki} %[section]

% Euron merkki
\usepackage{eurosym}

\begin{document}

\selectlanguage{finnish}

\begin{titlepage}

  \begin{center}
    \begin{doublespace}
      \begin{LARGE}
        \textrm{Hellsten -- Linja-aho -- Mauno -- Mäkinen -- Piiroinen -- Sottinen \ldots} \\
      \end{LARGE}
      
      \vspace{0.5cm}
      \hrule height 2pt
      \vspace{1cm}
      \begin{Huge}
        \textbf{\textrm{Avoin matikka 1}\\\ \\Kirja on työn alla!}
      \end{Huge}
      
      \vfill
      
      \begin{huge}
        \textrm{MAA1 -- Funktiot ja yhtälöt}
      \end{huge}
      \vspace{1cm}
      \hrule height 2pt
    \end{doublespace}
  \end{center}
  
  \vfill
  \begin{flushright}
    \textbf{Oppikirjamaraton - tätä lukee kuin avointa kirjaa! \\
      Sisältö on lisensoitu avoimella CC-BY-lisenssillä. \\
    }
  \end{flushright}
  
\end{titlepage}

\tableofcontents




%\part{Alku}
\chapter{Esipuhe}

%%%%%%%%%%%%%%%%%%%%%%%%%%%%%%%%%%%%%%%%%%%%%%%%%%%%%%%%%%%%%%%%%%%%%%%%%%%%%%%%
%%%%  /usr/share/doc/texlive-fonts-extra-doc/fonts/arev/mathtesty.tex

% mathtesty.tex, by Stephen Hartke 20050522
% based on mathtestx.tex in the mathptmx package
% and symbols.tex by David Carlisle

Lorem ipsum\ldots

\laatikko{Tässä on ältsin hieno teoriaboksi. Tänne voi laittaa myös kaavoja
\begin{equation}
(a+b)^2=a^2+2ab+b^2
\end{equation}
ja toimii kuin junan vessa.
}

\begin{theorem}[Residue Theorem]
Let $f$ be analytic in the region $G$ except for the isolated singularities $a_1,a_2,\ldots,a_m$. If $\gamma$ is a closed rectifiable curve in $G$ which does not pass through any of the points $a_k$ and if $\gamma\approx 0$ in $G$ then
\[
\frac{1}{2\pi i}\int_\gamma f = \sum_{k=1}^m n(\gamma;a_k) \text{Res}(f;a_k).
\]
\end{theorem}

\begin{esimerkki}[Leivän paino]
Leipä painaa kilon ja puolet leivästä. Painavako oli leipä?\\
{\bf Ratkaisu.} Merkitään leivän painoa $x$:llä. Puolet leivästä on matemaattisesti ilmaistuna $\frac{x}{2}$ ja kun siihen lisätään kilogramma, saadaan leivän paino, joten saamme yhtälön
\begin{equation}
\frac{x}{2}+1=x
\end{equation}
josta ratkeaa
\begin{equation}
x=2.
\end{equation}
Leipä painaa siis 2 kilogrammaa.
\end{esimerkki}

Another nice theorem from complex analysis is

\begin{theorem}[Maximum Modulus]
Let $G$ be a bounded open set in $\mathbb{C}$ and suppose that $f$ is a continuous function on $G^-$ which is analytic in $G$. Then
\[
\max\{|f(z)|:z\in G^-\}=\max \{|f(z)|:z\in \partial G \}.
\]
\end{theorem}

\newcommand{\abc}{abcdefgh\hbar\hslash i\imath j\jmath klmnopqrstuvwxyz}
\newcommand{\ABC}{ABCDEFGHIJKLMNOPQRSTUVWXYZ}
\newcommand{\alphabeta}{\alpha\beta\varbeta\gamma\delta\epsilon\varepsilon\zeta\eta\theta\vartheta\iota\kappa\varkappa\lambda\mu\nu\xi o\pi\varpi\rho\varrho\sigma\varsigma\tau\upsilon\phi\varphi\chi\psi\omega}
\newcommand{\AlphaBeta}{\Gamma\Delta\Theta\Lambda\Xi\Pi\Sigma\Upsilon\Phi\Psi\Omega}



%%%%%%%%%%%%%%%%%%%%%%%%%%%%%%%%%%%%%%%%%%%%%%%%%%%%%%%%%%%%%%%%%%%%%%%%%%%%%%%%
%%%% /usr/share/doc/texlive-doc-en/fonts/free-math-font-survey/source/textfragment.tex







%%% Local Variables: 
%%% mode: latex
%%% End: 




\part{Lukualueet}
\chapter{Luonnolliset luvut}

Tähän tekstiä luonnollisista luvuista.

\chapter{Joukko-oppia}
\chapter{Logiikkaa}
\chapter{Kokonaisluvut}
\chapter{Kokonaislukujen aritmetiikkaa}
\chapter{Jaollisuus \& tekijät}
\chapter{Rationaaliluvut ja laskusäännöt}

\begin{tehtava}
Laske %aika randomit luvut
\begin{enumerate}
\item $\frac{6}{2} + \frac{3}{5}$
\item $\frac{7}{8} - \frac{1}{4}$
\item $2 \frac{1}{3} + \frac{4}{6}$
\end{enumerate}
\begin{vastaus}
Vastaus:
\begin{enumerate}
\item $\frac{18}{5}$
\item $\frac{5}{8}$
\item $3$
\end{enumerate}
\end{vastaus}
\end{tehtava}

\chapter{Potenssisäännöt \& murtolausekkeiden sieventämistä}
Potenssilla $2^4$ tarkoitetaan tuloa $2\cdot 2\cdot 2\cdot 2$. Eli $2^4=2\cdot 2\cdot 2\cdot 2=16$. Lausekkeessa $2^4$ luku 2 on \textbf{kantaluku} ja luku 4 on \textbf{eksponentti}.

Vastaavalla logiikalla $5^3=5\cdot 5\cdot 5=125$ ja $(-7)^2=(-7)\cdot (-7)=49$.

\section{Potenssisäännöt}

\begin{tehtava}
%Pitkä Sigma 1, s.76
Laske lausekkeen arvo. Muista ottaa huomioon sulut!
\begin{enumerate}
\item $-2\cdot 4^2$
\item $(-2\cdot 4)^2$
\item $\frac{3^2}{7}$
\item $\left( \frac{3}{7} \right)^2$
\item $-2^6$
\item $(-2)^6$
\end{enumerate}
\begin{vastaus}
Vastaus:
\begin{enumerate}
\item $-32$
\item $64$
\item $\frac{9}{7}$
\item $\frac{9}{49}$
\item $-64$
\item $64$
\end{enumerate}
\end{vastaus}
\end{tehtava}

\begin{tehtava}
Sievennä
\begin{enumerate}
\item $x^7\cdot x^2$
\item $(a^3)^2$
\item $\frac{a^8}{a^2}$
\item $(2y+5x)^0$
\item $\frac{1}{10^{-3}}$
\end{enumerate}
\begin{vastaus}
Vastaus:
\begin{enumerate}
\item $x^9$
\item $a^6$
\item $a^4$
\item $1$
\item $1000$
\end{enumerate}
\end{vastaus}
\end{tehtava}

\section{Murtolausekkeiden sieventäminen}
%tässä pitää opettaa binomin neliösäännöt ja ne (3kpl)

\begin{tehtava}
%tää voi olla eka tehtävä
Sievennä
\begin{enumerate}
\item $\frac{a^2+2ab+b^2}{a+b}$
\item $\frac{a^2-2ab+b^2}{a-b}$
\item $\frac{a^2-b^2}{a+b}$
\end{enumerate}
\begin{vastaus}
Vastaus:
\begin{enumerate}
\item $a-b$
\item $a+b$
\item $a-b$
\end{enumerate}
\end{vastaus}
\end{tehtava}

\chapter{Juuret}

\section{Neliöjuuri}

\laatikko{Luvun $a$ neliöjuuri on ei-negatiivinen luku, jonka neliö on $a$. Tämä voidaan ilmaista lyhyemmin $\sqrt{b^2}=b$.}

Neliöjuuren määritteleminen $\sqrt{a}^2=a$ ei johda samaan lopputulokseen. Pohdi, miksi näin on.
%%vai parempi antaa suoraan $\sqrt{a}^2=a$, kun $a \ge 0$
Jatkossa tälaisia määritelmän pieniä muokkauksia ja niistä aiheutuvia muutoksia olisi aina hyvä pohdiskella -- saattavat jopa auttaa muistamaan määritelmän oikean muodon.
%%%%%%%%%%%%%%% ONKO ITSEISARVO KÄSITELTY!!!!! %%%%%%%%%%%%%%%%%%%%%%%%%
%Määtitelmäksi ei kelpaisi tämäkään. $\sqrt{a^2}=|a|$
%%%%%%%%%%%%%%% ONKO ITSEISARVO KÄSITELTY!!!!! %%%%%%%%%%%%%%%%%%%%%%%%%

Neliöjuurta ei siis nyt määritelty ollenkaan negatiivisille luvuille.

%yhtälöt tulevat vasta myöhemin, siksi esimerkit köyhiä

Esimerkki
\begin{align*}
\sqrt{4} = 2\ qquad\textrm{, koska $2>0$ ja $2^2 =4$} 
\end{align*}

%pythagoraan lause on pitänyt käydä ennen tätä!!
%tässä tehtävässä pitää osata tehdä ensimmäisen asteen yhtälö
Taulutelevision kooksi on ilmoitettu mainoksessa $46''$ ja kuvasuhteeksi 19:6. Kuinka leveä televisio on arviolta? ($1''$ = 1 tuuma = 2,54 cm)
Vastaus: $44''$ tai 111 cm

\section{Kuutiojuuri}

\laatikko{Luvun $a$ kuutiojuuri on luku, jonka kuutio on $a$. Tämä voidaan ilmaista lyhyemmin $\sqrt[3]{b^3}=b$.
Määritelmäksi voisi ottaa myös $\sqrt[3]{b^3}=b$.}

\section{n.s juuri}
Toista juurta $\sqrt[2]{a}$ merkitään $\sqrt{a}$


%\begin{Ex}
%Tehtävänanto
%\begin{solution}
%Ratkaisu
%\end{solution}
%\end{Ex}

\chapter{Murtopotenssi}
\chapter{Irrationaaliluvut}
\chapter{Reaaliluvut}
\chapter{Kompleksiluvut}
\chapter{Kertaustiivistelmä}

%
\part{Yhtälöt}
\chapter{Yhtälöiden teoriaa}
Monissa käytännön tilanteissa saamme samalle asialle kaksi erilaista esitystapaa.

\begin{esimerkki}
Meillä on orsivaaka, joka on tasapainossa. (kuva!) Toisessa vaakakupissa on kahden kilon siika ja toisessa puolen kilon ahven sekä tuntematon määrä lakritsia. Kuinka paljon vaakakupissa on lakritsia? (Ratkaistaan...) (Muita esimerkkejä, vähitellen vaikeutuvia (1. asteen) yhtälöitä)
\end{esimerkki}

Määritelmä: Yhtälöksi kutsutaan kahden lausekkeen merkittyä yhtäsuuruutta. Siis mielivaltaisille lausekkeille $A$ ja $B$ merkitään $A=B$. (Esim. $A=3x+5$ ja $B=7x+7$). Jos yhtälön puolien lausekkeiden arvot ovat samat, sanotaan että yhtälö pätee. 

Yhtälössä voi esiintyä myös muuttujia, eli symboleja joiden arvoa ei ole etukäteen määritelty. Muuttujia merkitään usein kirjaimilla $x$, $y$ ja $z$. Niitä muuttujien arvoja, joilla yhtälö pätee, kutsutaan yhtälön ratkaisuiksi. Yhtälön ratkaisemisella tarkoitetaan kaikkien yhtälön ratkaisujen selvittämistä.

Eräs tapa ratkaista yhtälöitä on muokata niitä niin, että muokattu yhtälö pätee täsmälleen silloin kun alkuperäinen yhtälö pätee. Tällaisia sallittuja muunnoksia ovat esimerkiksi:
\begin{itemize}
\item Yhtälön molemmat puolet voidaan kertoa nollasta poikkeavalla luvulla $m$. Muutos tehdään aina molemmille puolille. Tällöin saadaan yhtälö $mA = mB$.
\item Yhtälön molemmille puolille voidaan lisätä tai molemmilta puolilta vähentää luku $n$.
Tällöin saadaan yhtälö $A+n = B+n$.
\end{itemize}
[pitäskö perustella?]

Monet yhtälöt ratkeavat toistamalla tällaisia muunnoksia kunnes yhtälö on niin yksinkertaisessa muodossa, että ratkaisu on helppo nähdä. Koska jokaisessa muokkausjonon yhtälössä ratkaisut ovat samat, näin saadaan alkuperäisen yhtälön ratkaisut.

[joku esimerkki tähän?]

Yhtälöt voidaan ratkaisujensa perusteella jakaa kolmeen tyyppiin:
\begin{enumerate}
\item Yhtälö, joka on aina tosi. Esimerkiksi yhtälöt $8=8$ ja $x=x$.
\item Yhtälö, joka on joskus tosi. Esimerkiksi yhtälö $x+2=3$ on tosi jos ja vain jos $x=1$.
\item Yhtälö, joka ei ole koskaan tosi. Esimerkiksi yhtälö $0=1$.
\end{enumerate}
Tämän kurssin ja ylipäätään matematiikan kannalta selvästi tärkein yhtälötyyppi on 2. Siirrymme nyt tarkastelemaan tärkeää erikoistapausta yhtälöistä, ensimmäisen asteen yhtälöitä.

\chapter{Ensimmäisen asteen yhtälö}
Ensimmäisen asteen yhtälö on yhtälö, joka on esitettävissä muodossa $ax+b=0$, jossa $a \neq 0$.

\begin{theorem}
Kaikki muotoa $ax+b=cx+d$ olevat yhtälöt, joissa $a \neq c$, ovat ensimmäisen asteen yhtälöitä.
\end{theorem}

\begin{proof}
\begin{align*}
ax+b &= cx+d & &| \, \textbf{Vähennetään molemmilta puolilta $cx+d$.} \\
ax+b - (cx+d) &= 0 & &| \, 
\end{align*}
\end{proof}

\begin{theorem}
Yleinen lähemistymistapa muotoa $ax+b = cx+d$ olevien yhtälöiden ratkaisuun on: \\
(1) Vähennä molemmilta puolilta $cx$. Saat yhtälön $(a-c)x + b = d$. \\
(2) Vähennä molemmilta puolita $b$. Saat yhtälön $(a-c)x = d-b$. \\
(3) Jaa $(A-C)$:llä. Saat yhtälön ratkaistuun muotoon $x = \frac{d-b}{a-c}$.
\end{theorem}

Esimerkki. Yhtälön $7x+4=4x+7$ ratkaisu saadaan seuraavasti:
\begin{align*}
7x+4 &= 4x+7 & &| \, \text{Vähennetään molemmilta puolilta 4x.} \\
3x+4 &= 7 & &| \, \text{Vähennetään molemmilta puolilta 4.} \\
3x &= 3 & &| \, \text{Jaetaan molemmat puolet kolmella eli kerrotaan $\frac{1}{3}$:lla.} \\
x &= 1 & &| \, \text{Saimme yhtälön ratkaistuun muotoon. $x=1$ on siis yhtälön ratkaisu.} \\
\end{align*}

\begin{tehtava}
%
Ratkaise:
\begin{enumerate}
\item $x + 4 = 5$
\item $1 - x = -3$
\item $7x = 35$
\item $-2x = 4$
\item $10 - 2x = x$
\item $9x + 4 = 6 - x$
\item $\frac{2x}{5} = 4$
\item $\frac{x}{3} + 1 = \frac{5}{6} - x$

\end{enumerate}

\end{tehtava}

\chapter{Yhtälöpari}


\chapter{Yleinen potenssi ja potenssiyhtälö}
%pitää esitellä mitä on mega, milli, sentti jne.

\begin{tehtava}
Esitä luku ilman kymmenpotenssia.
\begin{enumerate}
\item $3,2 * 10^4$
\item $-7,03 * 10^{-5}$
\item $10,005 * 10^{-2}$
\end{enumerate}
\begin{vastaus}
Vastaus
\begin{enumerate}
\item $32000$
\item $-0,0000703$
\item $0,10005$
\end{enumerate}
\end{vastaus}
\end{tehtava}

\begin{tehtava}
Esitä luku ilman etuliitettä.
\begin{enumerate}
\item $0,5 dl$
\item $233 mm$
\item $33 cm$
\item $16 kg$
\item $2 MJ$
\item %megatavu, mibitavu jne.
\item 
\end{enumerate}
\begin{vastaus}
Vastaus:
\begin{enumerate}
\item $0,05 l$
\item $0,233 m$
\item $0,33 m$
\item $16 000 g$
\item $2 000 000 J$
\item $ $
\item $ $
\end{enumerate}
\end{vastaus}
\end{tehtava}


\chapter{Kertaustiivistelmä}

\part{Funktiot}
\chapter{Funktio}
Matematiikassa tutkitaan paljon suureiden välisiä riippuvuuksia. Tällaiset riippuvuudet voidaan muotoilla funktioiden avulla. Esimerkiksi tuotteen arvonlisäveroprosentti riippuu tuotteen tyypistä. Tämä riippuvuus voidaan kirjoittaa funktiona eri tuotetyyppien joukolta $A$ reaalilukujen joukolle $\mathbb{R}$, missä funktio liittää jokaiseen tuotteeseen sen arvonlisäveroprosentin.

[Esimerkki, kuva arvonlisäverofunktiosta, missä \[A = \{\text{ahvenfilee}, \text{AIV-rehu}, \text{auto}, \text{runokirja}, \text{ravintola-ateria}, \text{särkylääke}, \text{televisio}\},\]$f(\text{ahvenfilee}) = 13$, $f(\text{AIV-rehu}) = 13$, $f(\text{auto}) = 23$, $f(\text{runokirja}) = 9$, $f(\text{ravintola-ateria}) = 13$, $f(\text{särkylääke}) = 9$, $f(\text{televisio}) = 23$]


\laatikko{Funktio $f$ joukosta $A$ joukkoon $B$ on sääntö, joka liittää $A$:n jokaiseen alkioon täsmälleen yhden $B$:n alkion.}


\chapter{Erilaisia funktioita}

%
\part{Sovelluksia}
%
%(Pythagoraan lause)
%
\chapter{Verrannollisuus}

Verrannollisuus
Verrannollisuus: sovelluksia
Prosenttilaskentaa – perustilanteet

promille, osuuden ottaminen ja osuuden/suhteen laskeminen, muutosprosentti

vertailuprosentti
Prosenttiyhtälöitä ja sovelluksia

yhtälöitä, sovelluksia; ALV, vero, hintojen muutos ei sama... (paljon     esimerkkejä yleisistä väärinymmärryksistä!!)
(Eksponentiaalinen malli)
Kertaustiivistelmä


%
\part{Kertaus ja harjoituskokeita}
\chapter{Verrannollisuus}

    Kertausosio (teoria ja esimerkit)
    Kertaustehtäväsarjoja
    Harjoituskokeita
    “Näihin pystyt jo” -yo-tehtäviä (myös lyhyestä)
    “Näihin pystyt jo” -pääsykoetehtäviä (moooonilta eri     aloilta! kauppatieteellinen, tradenomi (jos löytyy), kansantaloustiede, arkkitehtuuri, DI-haku, AMK tekniikan alat, fysiikka, tilastotiede, ...)
    Vastauksia ja ratkaisuja
    Suomi-ruotsi-englanti-sanasto ja hakemisto
    symbolitaulukko

% Tähän tulee liitteitä
% Esimerkiksi loogiset symbolit, reaalilukujen aksioomat, kompleksilukuintro, ...
\part{Liiteet}
% Vaihda tähän kirjaimin kulkeva "numerointi"
\chapter{Logiikka ja joukko-oppi}
\chapter{Reaalilukujen aksioomat}
Reaaliluvut ovat kunta, eräs algebrallinen rakenne. Myös esimerkiksi rationaaliluvut ja seuraavassa liitteessä esiteltävät kompleksiluvut muodostavat kunnan. Sen sijaan luonnolliset luvut ja kokonaisluvut eivät ole kuntia.

Reaalilukujen aksiomaattinen määritelmä muodostuu kolmesta osasta:

\textbf{1. Kunta-aksioomat reaalilukuihin sovellettuna} \\
\begin{align*}
&\text{K1.} \, \forall x, y \in \mathbb{R}: x+(y+z) = (x+y)+z & &| \, \text{summan liitäntälaki} \\
&\text{K2.} \, \exists 0 \in \mathbb{R}: x+0 = x & &| \, \text{summan neutraalialkio} \\
&\text{K3.} \, \forall x \in \mathbb{R} \, \exists (-x) \in \mathbb{R}: x+(-x)=0 & &| \, \text{vasta-alkio} \\
&\text{K4.} \, \forall x, y \in \mathbb{R}: x+y = y+x & &| \, \text{summan vaihdantalaki} \\
&\text{K5.} \, \forall x, y, z \in \mathbb{R}: x*(y+z) = x*y + x*z & &| \, \text{osittelulaki} \\
&\text{K6.} \, \forall x, y, z \in \mathbb{R}: x*(y*z) = (x*y)*z & &| \, \text{tulon liitäntälaki} \\
&\text{K7.} \, \exists 1 \in \mathbb{R}: 1*x = x & &| \, \text{tulon neutraalialkio} \\
&\text{K8.} \, \forall x \in \mathbb{R} \setminus \{0\} \, \exists x^{-1} \in \mathbb{R} \setminus \{0\}: x*x^{-1}=1 & &| \, \text{tulon käänteisalkio} \\
&\text{K9.} \, \forall x, y \in \mathbb{R}: x*y = y*x & &| \, \text{tulon vaihdantalaki}
\end{align*}
\end{document}

% -*- coding: utf-8 -*-

%%% Local Variables: 
%%% mode: latex
%%% TeX-master: t
%%% coding: utf-8
%%% End: 

